%%%%%%%%%%%%%%%%%%%%%%%%%%%%%%%%%%%%%%%%%%%%%%%%%%%%%%%%%%%%%
%% Isolated DC-DC converters %%
%%%%%%%%%%%%%%%%%%%%%%%%%%%%%%%%%%%%%%%%%%%%%%%%%%%%%%%%%%%%%
\section{Isolated DC-DC converters}

%%%%%%%%%%%%%%%%%%%%%%%%%%%%%%%%%%%%%%%%%%%%%%%%%%%%%%%%%%%%%
%% Some fundamentals %%
%%%%%%%%%%%%%%%%%%%%%%%%%%%%%%%%%%%%%%%%%%%%%%%%%%%%%%%%%%%%%
\subsection{Some fundamentals}

%%%%%%%%%%%%%%%%%%%%%%%%%%%%%%%%%%%%%%%%%%%%%%%%%%%%%%%%%%%%%
%% Galvanic isolation %%
%%%%%%%%%%%%%%%%%%%%%%%%%%%%%%%%%%%%%%%%%%%%%%%%%%%%%%%%%%%%%
\begin{frame}
    \frametitle{Galvanic isolation}
    \begin{columns}
        \begin{column}{0.5\textwidth}
             \begin{varblock}{A definition}
                Galvanic isolation is a principle of decoupling functional sections of electrical circuits to prevent a direct current flow from input to output, that is, enabling different ground potentials for the circuit sections.   
             \end{varblock}%
             \vspace{1em}
        \onslide<2->{Typical reasons for requiring galvanic isolation are:}
        \begin{itemize}
            \item<2-> Safety (prevention of electric shock),
            \item<3-> Noise reduction,
            \item<4-> contact corrosion reduction.
        \end{itemize}
        \end{column}
        \begin{column}{0.5\textwidth}
            \vspace{-0.75cm}
            \begin{figure}
                \begin{subfigure}{\textwidth} 
                    \begin{circuitikz}
                        \ctikzset{blocks/scale=2, block lateral anchors pos=0.7}
                        \path (0,0) node[twoportsplitshape](dcdc){} ; 
                        \draw (dcdc.left up) -- ++(-1,0) coordinate (A1) 
                        (dcdc.left down) to [short, -*] ++(-1,0) coordinate (A2)
                        (A1) to [V, v_=$u_1$] (A2)
                        (dcdc.right up) to [short, -o]  ++(1,0) coordinate (B1)
                        (dcdc.right down) to [short, -o] ++(1,0) coordinate (B2);
                        \draw (B1) to [open, v^=$u_2$, voltage = straight] (B2);
                        \path (3.25,0) node[graduate,mirrored,sword,minimum size=1.5cm, stripes = uniblue] (human){};
                        \draw (human.south) node[ground](G2){}
                        let \p1 = (G2), \p2 = (A2) in (\x2, \y1) node[ground](G1){}
                        (G1) -- (A2);
                        \draw let \p1 = (G2), \p2 = (A2)  in (\x2, \y1) [signaldelta, ultra thick, -latex, dashed] (G1.south) -- (\x2,\y2) -- (\x1,\y2) -- (G2.south) -- (G1.south);
                        \draw let \p1 = (G2), \p2 = (A2)  in (\x2/2+\x1/2, \y1) node[signaldelta, below] {current through ground};
                    \end{circuitikz}
                \caption{Lack of galvanic isolation}
            \end{subfigure}
            \onslide<2->{%
            \begin{subfigure}{\textwidth} 
                \begin{circuitikz}
                    \ctikzset{blocks/scale=2, block lateral anchors pos=0.7}
                    \path (0,0) node[twoportsplitshape](dcdc){} ; 
                    \draw (dcdc.left up) -- ++(-1,0) coordinate (A1) 
                    (dcdc.left down) to [short, -*] ++(-1,0) coordinate (A2)
                    (A1) to [V, v_=$u_1$] (A2)
                    (dcdc.right up) to ++(0.15,0) node[transformer, anchor=A1, scale=0.655](T){}
                    (T.A2) -- (dcdc.right down)
                    (T.B1) to [open, v^=$u_2$, voltage = straight, o-o] (T.B2);
                    \path (3.7,0) node[graduate,mirrored,sword,minimum size=1.5cm, stripes = uniblue] (human){};
                    \draw (human.south) node[ground](G2){}
                    let \p1 = (G2), \p2 = (A2) in (\x2, \y1) node[ground](G1){}
                    (G1) -- (A2);
                    \draw [signalbeta, ultra thick, latex-latex, dashed] (G1.south) (G2.south) -- (G1.south);
                    \draw let \p1 = (G2), \p2 = (A2)  in (\x2/2+\x1/2, \y1) node[signalbeta, below] {output $u_2$ can 'float'};
                \end{circuitikz}
                \caption{Galvanic isolation via inductive separation}
            \end{subfigure}
            }%
            \caption{Why galvanic isolation can be useful}
            \label{fig:galvanic-isolation}
            \end{figure}
        \end{column}
    \end{columns}
\end{frame}

%%%%%%%%%%%%%%%%%%%%%%%%%%%%%%%%%%%%%%%%%%%%%%%%%%%%%%%%%%%%%
%% Galvanic isolation: technical realization %%
%%%%%%%%%%%%%%%%%%%%%%%%%%%%%%%%%%%%%%%%%%%%%%%%%%%%%%%%%%%%%
\begin{frame}
    \frametitle{Galvanic isolation: technical realization}
    \begin{table}
		\centering
		\begin{tabular}{M{0.3\textwidth} M{0.3\textwidth} M{0.3\textwidth}}
			\onslide<1->{Capacative} & \onslide<2->{Optical} & \onslide<3->{Inductive}\\[1em]

			\onslide<1->{
			\begin{circuitikz}
				\draw (0,-0.5) to [capacitor] (2,-0.5)
                (0,0.5) to [capacitor] (2,0.5);
			\end{circuitikz}
			}

			&
			\onslide<2->{
			\begin{circuitikz}
				\draw (0,0.5) to [empty photodiode, mirror] (2,0.5)
                (0,-0.5) to [empty led] (2,-0.5);
			\end{circuitikz}
			}
			
			&
			\onslide<3->{
			\begin{circuitikz}
				\draw (0,0) node[transformer](){};
			\end{circuitikz}
			}
            
            \\[1em]  
            
            \includegraphics[height=0.25\textheight]{fig/lec03/Capacitor_example.jpg}
            
            {\small source: \href{https://commons.wikimedia.org/wiki/File:Plattenkondensator_hg.jpg}{Wikimedia Commons}, H.~Grobe, \href{https://creativecommons.org/licenses/by/3.0/deed.en}{CC~BY~3.0}}

            &

            \onslide<2->{
            \includegraphics[height=0.25\textheight]{fig/lec03/Optocoupler_example.jpg}

            {\small source: \href{https://en.wikipedia.org/wiki/File:Philips_BDP3280-12_-_Everlight_EL817_on_power_board-1779.jpg}{Wikimedia Commons}, R.~Spekking, \href{https://creativecommons.org/licenses/by-sa/4.0/deed.en}{CC~BY-SA~4.0}}
            }

            &

            \onslide<3->{
            \includegraphics[height=0.25\textheight]{fig/lec03/Transformer_example.jpg}

            {\small source: \href{https://commons.wikimedia.org/wiki/File:Trafo-innenleben.jpg}{Wikimedia Commons}, S.~Riepl, public domain}
            }

		\end{tabular}
	\end{table}
\end{frame}

%%%%%%%%%%%%%%%%%%%%%%%%%%%%%%%%%%%%%%%%%%%%%%%%%%%%%%%%%%%%%
%% Galvanic isolation via transformer %%
%%%%%%%%%%%%%%%%%%%%%%%%%%%%%%%%%%%%%%%%%%%%%%%%%%%%%%%%%%%%%
\begin{frame}
    \frametitle{Galvanic isolation via transformer}
    \begin{columns}
        \begin{column}{0.4\textwidth}
            \begin{itemize}
                \item In power electronics, transformers are mostly used to provide galvanic isolation.
                \item<2-> Reason: the power density per volume and weight is typically higher than for capacitive or optical isolation.
                \item<3-> Assumptions for the following model:
                \begin{itemize}
                    \item<3-> Ideal coupling \newline(no leakage flux),
                    \item<4-> no losses,
                    \item<5-> no saturation.
                \end{itemize}
            \end{itemize}
        \end{column}
        \begin{column}{0.6\textwidth}
            \begin{figure}
                \begin{tikzpicture}[global scale/.style={scale=1.0}, rotate=-5, xslant=-0.1, thick, every
                    node/.style={transform shape, scale=0.8}, decoration={markings, mark=at position 0.5 with {\arrow{latex}}}]
                    \tikzmath{
                        real \a, \b, \dx, \dy, \lx, \ly, \dr;
                        \a = 3.0;
                        \b = 5.0;
                        \dx = 0.8;
                        \dy = 0.5;
                        \lx = 1.0;
                        \ly = 1.0;
                        \dr = 0.02;
                    } 
                    \begin{scope}[even odd rule, scale=0.9]
                    \filldraw[rounded corners=2pt, fill=gray, rotate=-0, opacity=1.0] (\dx,
                    \dy) rectangle ++(5,5) (\lx+\dx,\ly+\dy) rectangle ++(\a, \a);
                    \fill [rounded corners=2pt, fill=gray] (\b, 0) --++ (0, \dy+\dr+0.02) --++(\dx, 0) --cycle;
                    \fill [rounded corners=2pt, fill=gray] (0, \b) --++ (\dx+\dr+0.02, 0) --++(0, \dy)--cycle;
                    \filldraw[rounded corners=2pt, fill=gray!50, rotate=-0] (0,0) rectangle
                    ++(\b, \b) (\lx,\ly) rectangle ++(\a, \a);
                    \draw (\b-\dr,\dr) --++(\dx, \dy);
                    \draw (\b-\dr,\b-\dr) --++(\dx, \dy);
                    \draw (\dr,\b-\dr) --++(\dx, \dy);
                    \draw [signalalpha, thick, postaction={decorate}] (0, \ly) --++(-1.5,0);
                    \draw [rounded corners=2pt,signalalpha, thick]
                        (-0.09,\ly) -- (\lx, \ly)--++(0.89,0.5)
                        --++(-0.08, 0.05);
                    \foreach \z in {.24,.48,...,2.5}
                        {
                        \draw [rounded corners=2pt,signalalpha, thick]
                        (-0.0,\ly+\z+0.08)--(-0.09,\ly+\z) -- (\lx, \ly+\z)--++(0.89,0.5)
                        --++(-0.08, 0.05);
                    }
                    \draw [rounded corners=2pt,signalalpha, thick, postaction={decorate}] (-1.5,
                    \ly+2.8) --++(1.5,0) node[black, above, pos=0.4] {$i_1(t)$};
                    \draw[latex-] (-\lx, \ly+0.2) --++(0, 2.4) node[midway, left] {$u_1(t)$};
                    
                    \draw [rounded corners=2pt,signaldelta, thick] (\a+\lx-2*\dr,
                    \ly+2*\dr)--++(-\dr, -\dr)--++(\lx+\dx+2*\dr, 0);
                    \draw [signaldelta, postaction={decorate}] (\b+\dx-\dr+\a/2, \ly+\dr)--(\b+\dx-\dr, \ly+\dr);
                    \draw [rounded corners=2pt, signaldelta, thick, postaction={decorate}]
                    (\a+3*\lx-0.2, \ly+3.1) --++(\a/2, 0) node[black, midway, above] {$i_2(t)$};
                    \foreach \z in {.125,.25,.375,...,2.5}
                        {
                        \draw [rounded corners=2pt, signaldelta, thick] (\a+\lx,\ly+\z+0.1)--
                        (\a+\lx-0.07,\ly+\z) -- (\a+2*\lx, \ly+\z)--++(0.87,0.5)--++(-0.06,
                        0.06);
                        }
                    \draw[latex-] (2*\a+\lx, \ly+0.25) --++(0, 2.6) node[midway, left] {$u_2(t)$};
                    \end{scope}
                \end{tikzpicture}
                \caption{Simple transformer with primary and secondary winding}
                \label{fig:transformer_pseudo_3D}
            \end{figure}
        \end{column}
    \end{columns}
\end{frame}

%%%%%%%%%%%%%%%%%%%%%%%%%%%%%%%%%%%%%%%%%%%%%%%%%%%%%%%%%%%%%
%% Simplistic transformer model %%
%%%%%%%%%%%%%%%%%%%%%%%%%%%%%%%%%%%%%%%%%%%%%%%%%%%%%%%%%%%%%
\begin{frame}
    \frametitle{Simplistic transformer model}
    \begin{figure}
        \begin{subfigure}{0.4\textwidth}
            \centering
            \begin{tikzpicture}
                \draw (0,0) node[transformer core](T){$N_1:N_2$}
                (T.inner dot A1) node[circ]{}
                (T.inner dot B1) node[circ]{}
                (T.A1) to [short, -o, i<_=$i_1(t)$] ++(-1,0) coordinate (A1)
                (T.A2) to [short, -o] ++(-1,0) coordinate (A2)
                (T.B1) to [short, -o, i=$i_2(t)$] ++(1,0) coordinate (B1)
                (T.B2) to [short, -o] ++(1,0) coordinate (B2);
                \draw (A1) to [open, v=$u_1$, voltage = straight] (A2)
                (B1) to [open, v^=$u_2$, voltage = straight] (B2);
                \draw[dashed, thin] let \p1=(T.A1), \p2=(T.B2) in (\x1 + 2mm,\y1 + 10mm) rectangle (\x2 - 2mm,\y2 - 12.5mm)
                (\x1/2 + \x2/2,\y1 + 10mm) node[above] {Transformer};
            \end{tikzpicture}
           \caption{Schematic symbol representation} 
        \end{subfigure}
        \onslide<2->{%
            \begin{subfigure}{0.4\textwidth}
                \centering
                \begin{tikzpicture}
                    \draw (0,0) node[transformer core](T){$N_1:N_2$}
                    (T.inner dot A1) node[circ]{}
                    (T.inner dot B1) node[circ]{}
                    (T.A1) to [short, -*, i<_=$i'_1(t)$] ++(-1,0) coordinate (A1)
                    (T.A2) to [short, -*] ++(-1,0) coordinate (A2)
                    (A1) to [inductor, l=$L_\mathrm{m}$, i=$i_\mathrm{m}(t)$] (A2)
                    (A1) to [short, -o, i<_=$i_1(t)$] ++(-1.5,0) coordinate (A11)
                    (A2) to [short, -o] ++(-1.5,0) coordinate (A22)
                    (T.B1) to [short, -o, i=$i_2(t)$] ++(1,0) coordinate (B1)
                    (T.B2) to [short, -o] ++(1,0) coordinate (B2);
                    \draw (A11) to [open, v=$u_1$, voltage = straight] (A22)
                    (B1) to [open, v^=$u_2$, voltage = straight] (B2);
                    \draw[dashed, thin] let \p1=(A1), \p2=(T.B2) in (\x1 - 3mm,\y1 + 10mm) rectangle (\x2,\y2 - 12.5mm)
                    (\x1/2 + \x2/2 -2.5mm,\y1 + 10mm) node[above] {Transformer};
                    \draw[dashed, thin] let \p1=(T.A1), \p2=(T.B2) in (\x1 + 2mm,\y1 + 5mm) rectangle (\x2 - 2mm,\y2 - 2.5mm)
                    (\x1/2 + \x2/2,\y2 - 2.5mm) node[below, align=center] {Ideal\\ transformer};
                \end{tikzpicture}
                \caption{Equivalent circuit model} 
            \end{subfigure}
        }%
        \caption{Transformer model}
        \label{fig:transformer_model}
    \end{figure}
\end{frame}

%%%%%%%%%%%%%%%%%%%%%%%%%%%%%%%%%%%%%%%%%%%%%%%%%%%%%%%%%%%%%
%% Simplistic transformer model (cont.) %%
%%%%%%%%%%%%%%%%%%%%%%%%%%%%%%%%%%%%%%%%%%%%%%%%%%%%%%%%%%%%%
\begin{frame}
    \frametitle{Simplistic transformer model (cont.)}
    Based on \figref{fig:transformer_model} we consider the transformer as a combination of an ideal transformer with the \hl{conversion ratios}
    \begin{equation}
        \frac{u_1(t)}{u_2(t)} = \frac{N_1}{N_2} \quad \text{and} \quad \frac{i'_1(t)}{i_2(t)} = \frac{N_2}{N_1}
    \end{equation}\pause%
    and an inductor with the \hl{magnetizing inductance} $L_\mathrm{m}$:
    \begin{equation}
        u_1(t) = L_\mathrm{m} \frac{\mathrm{d}i_\mathrm{m}(t)}{\mathrm{d}t} \quad \text{and} \quad i_1(t) = i'_1(t) + i_\mathrm{m}(t).
    \end{equation}\pause %
    \begin{itemize}
        \item<3-> $L_\mathrm{m}$ models the magnetic energy stored in the transformer.
        \item<4-> Above model is a significant simplification (very first principle approach).
        \item<5-> More details on the transformer model can be found in the \href{https://github.com/IAS-Uni-Siegen/EMD_course}{Electrical Machines and Drives} course material.
    \end{itemize}
\end{frame}

%%%%%%%%%%%%%%%%%%%%%%%%%%%%%%%%%%%%%%%%%%%%%%%%%%%%%%%%%%%%%
%% Flyback converter %%
%%%%%%%%%%%%%%%%%%%%%%%%%%%%%%%%%%%%%%%%%%%%%%%%%%%%%%%%%%%%%
\subsection{Flyback converter}

%%%%%%%%%%%%%%%%%%%%%%%%%%%%%%%%%%%%%%%%%%%%%%%%%%%%%%%%%%%%%
%% Topology derivation based on the inverting buck-boost converter %%
%%%%%%%%%%%%%%%%%%%%%%%%%%%%%%%%%%%%%%%%%%%%%%%%%%%%%%%%%%%%%
\begin{frame}
    \frametitle{Topology derivation based on the inverting buck-boost converter}
    \begin{figure}
        \begin{circuitikz}[]
            \draw (0,0) to [short, *-] ++(1.0,0)
            to [diode, l=$D$, invert]  ++(1.0,0)
            to [short, -o, i=$i_2(t)$] ++(1.0,0)
            to [open, o-o, v^= $u_2(t)$, voltage = straight] ++(0,-2) coordinate (A)
            (0,0) to [short, *-] ++(-1.0,0) 
            to [Tnpn, n=npn1] ++(-1.0,0)
            to [short, -o, i_<=$i_1(t)$] ++(-1.0,0)
            to [open, o-o, v_= $u_1(t)$, voltage = straight] ++(0,-2)
            to [short, o-o] (A)
            (0,0) to [inductor, *-*] ++(0,-2);
            \draw let \p1 = (npn1.B) in node[anchor=north] at (\x1,\y1) {$T$};
            \draw [decorate,decoration={brace,amplitude=10pt,mirror,raise=0.5cm},yshift=0pt] (-3,-2.0) -- (3,-2.0) node [black,midway,yshift=-0.6cm] {};
        \end{circuitikz}
        \begin{circuitikz}[]
            \draw (0,0) to [short] ++(1.0,0)
            to [diode, l=$D$, invert]  ++(1.0,0)
            to [short, -o, i=$i_2(t)$] ++(1.0,0)
            to [open, o-o, v^= $u_2(t)$, voltage = straight] ++(0,-2) coordinate (A)
            (0,0) to [short] ++(-1.0,0) 
            to [Tnpn, n=npn1] ++(-1.0,0)
            to [short, -o, i_<=$i_1(t)$] ++(-1.0,0)
            to [open, o-o, v_= $u_1(t)$, voltage = straight] ++(0,-2)
            to [short, o-o] (A)
            (-0.5,0) to [inductor, *-*, n=l1] ++(0,-2)
            (0.5,0) to [inductor, *-*, n=l2, mirror] ++(0,-2);
            \draw let \p1 = (npn1.B) in node[anchor=north] at (\x1,\y1) {$T$};
            \path (l1.ul dot) node[circ]{}
                (l2.ul dot) node[circ]{};
            \draw (l1.midtap) node[left]{$N_1$}
            (l2.midtap) node[right]{$N_2$};
            \draw[double, double distance=3pt, thick] let \p1=(l1.core west), \p2=(l2.core east) in (\x1/2+\x2/2, \y1) -- (\x1/2+\x2/2, \y2);
        \end{circuitikz}
    \end{figure}
\end{frame}

%%%%%%%%%%%%%%%%%%%%%%%%%%%%%%%%%%%%%%%%%%%%%%%%%%%%%%%%%%%%%
%% Topology derivation based on the inverting buck-boost converter (cont.) %%
%%%%%%%%%%%%%%%%%%%%%%%%%%%%%%%%%%%%%%%%%%%%%%%%%%%%%%%%%%%%%
\begin{frame}
    \frametitle{Topology derivation based on the inverting buck-boost converter (cont.)}
    \begin{figure}
        \begin{circuitikz}[]
            \draw (0,0) to [short] ++(1.0,0)
            to [diode, l=$D$, invert]  ++(1.0,0)
            to [short, -o, i=$i_2(t)$] ++(1.0,0)
            to [open, o-o, v^= $u_2(t)$, voltage = straight] ++(0,-2) coordinate (A)
            (0,0) to [short] ++(-1.0,0) 
            to [Tnpn, n=npn1] ++(-1.0,0)
            to [short, -o, i_<=$i_1(t)$] ++(-1.0,0)
            to [open, o-o, v_= $u_1(t)$, voltage = straight] ++(0,-2)
            to [short, o-o] (A)
            (-0.5,0) to [inductor, *-*, n=l1] ++(0,-2)
            (0.5,0) to [inductor, *-*, n=l2, mirror] ++(0,-2);
            \draw let \p1 = (npn1.B) in node[anchor=north] at (\x1,\y1) {$T$};
            \path (l1.ul dot) node[circ]{}
                (l2.ul dot) node[circ]{};
            \draw (l1.midtap) node[left]{$N_1$}
            (l2.midtap) node[right]{$N_2$};
            \draw[double, double distance=3pt, thick] let \p1=(l1.core west), \p2=(l2.core east) in (\x1/2+\x2/2, \y1) -- (\x1/2+\x2/2, \y2);
            \draw [decorate,decoration={brace,amplitude=10pt,mirror,raise=0.5cm},yshift=0pt] (-3,-2.0) -- (3,-2.0) node [black,midway,yshift=-0.6cm] {};
        \end{circuitikz}
        \begin{circuitikz}[]
            \draw (0.5,0) to [short] ++(0.5,0)
            to [diode, l=$D$, invert]  ++(1.0,0)
            to [short, -o, i=$i_2(t)$] ++(1.0,0)
            to [open, o-o, v^= $u_2(t)$, voltage = straight] ++(0,-2) coordinate (A)
            (-0.5,0) to [short] ++(-0.5,0) 
            to [Tnpn, n=npn1] ++(-1.0,0)
            to [short, -o, i_<=$i_1(t)$] ++(-1.0,0)
            to [open, o-o, v_= $u_1(t)$, voltage = straight] ++(0,-2) coordinate (B)
            (-0.5,0) to [inductor, n=l1] ++(0,-2) coordinate (C)
            (0.5,0) to [inductor, n=l2, mirror] ++(0,-2) coordinate (D)
            (D) to [short, -o] (A)
            (C) to [short, -o] (B);
            \draw let \p1 = (npn1.B) in node[anchor=north] at (\x1,\y1) {$T$};
            \path (l1.ul dot) node[circ]{}
                (l2.ul dot) node[circ]{};
            \draw (l1.midtap) node[left]{$N_1$}
            (l2.midtap) node[right]{$N_2$};
            \draw[double, double distance=3pt, thick] let \p1=(l1.core west), \p2=(l2.core east) in (\x1/2+\x2/2, \y1) -- (\x1/2+\x2/2, \y2);
        \end{circuitikz}
    \end{figure}
\end{frame}

%%%%%%%%%%%%%%%%%%%%%%%%%%%%%%%%%%%%%%%%%%%%%%%%%%%%%%%%%%%%%
%% Flyback converter: topology %%
%%%%%%%%%%%%%%%%%%%%%%%%%%%%%%%%%%%%%%%%%%%%%%%%%%%%%%%%%%%%%
\begin{frame}
    \frametitle{Flyback converter: topology}
    \begin{columns}
        \begin{column}{0.5\textwidth}
            \begin{itemize}
                \item \hl{Flyback converter = non-inverting, galvanically isolated buck-boost converter}.
                \item<2-> Polarity change of primary and secondary transformer windings compensate for the inverting buck-boost characteristic.
                \item<3-> Transistor $T$ is placed below the transformer to enable a fixed emitter / source potential (beneficial for driver).
                \item<4-> Transformer's magnetizing inductance serves as the converter's energy buffer.
            \end{itemize}
        \end{column}
        %
        \begin{column}{0.5\textwidth}
            \begin{figure}
                \begin{circuitikz}[]
                    \draw (0.5,0) to [short] ++(0.5,0)
                    to [diode, l=$D$]  ++(1.0,0)
                    to [short, -o, i=$i_2(t)$] ++(1.0,0)
                    to [open, o-o, v^= $u_2(t)$, voltage = straight] ++(0,-2) coordinate (A)
                    (-0.5,0) to [short, -o, i_<=$i_1(t)$] ++(-1.5,0)
                    to [open, o-o, v_= $u_1(t)$, voltage = straight] ++(0,-3.75) coordinate (B)
                    (-0.5,0) to [inductor, n=l1] ++(0,-2) 
                    to [Tnpn, n=npn1, invert] ++(0,-1.75) coordinate (C)
                    (0.5,0) to [inductor, n=l2, mirror] ++(0,-2) coordinate (D)
                    (D) to [short, -o] (A)
                    (C) to [short, -o] (B);
                    \draw let \p1 = (npn1.B) in node[anchor=south] at (\x1,\y1) {$T$};
                    \path (l1.ul dot) node[circ]{}
                        (l2.ur dot) node[circ]{};
                    \draw (l1.midtap) node[left]{$N_1$}
                    (l2.midtap) node[right]{$N_2$};
                    \draw[double, double distance=3pt, thick] let \p1=(l1.core west), \p2=(l2.core east) in (\x1/2+\x2/2, \y1) -- (\x1/2+\x2/2, \y2);
                \end{circuitikz}
                \caption{Flyback converter topology}
                \label{fig:flyback_converter_topology}
            \end{figure}
        \end{column}
    \end{columns}
\end{frame}

%%%%%%%%%%%%%%%%%%%%%%%%%%%%%%%%%%%%%%%%%%%%%%%%%%%%%%%%%%%%%
%% Flyback converter: switching states %%
%%%%%%%%%%%%%%%%%%%%%%%%%%%%%%%%%%%%%%%%%%%%%%%%%%%%%%%%%%%%%
\begin{frame}
    \frametitle{Flyback converter: switching states in CCM}
    \begin{itemize}
        \item Switch-on time: rising primary current induces a negative voltage at the transformer's secondary winding leading to blocking diode. Energy is stored in $L_\mathrm{m}$.
        \item<2-> Switch-off time: primary current is blocked by transistor and an equivalent current is induced in the secondary winding. Energy is taken from $L_\mathrm{m}$.
    \end{itemize}
    \begin{figure}
        \begin{subfigure}{0.45\textwidth}
            \centering
            \begin{circuitikz}[]
                \draw (0.5,0) to [short] ++(0.5,0)
                to [open]  ++(1.0,0)
                to [short, -o, i={$i_2(t)=0$}] ++(1.0,0)
                to [open, o-o, v^= $u_2(t)$, voltage = straight] ++(0,-2) coordinate (A)
                (-0.5,0) to ++(-1,0) coordinate (E)
                to [short, -o, i_<=$i_1(t)$] ++(-1.0,0)
                to [open, o-o, v_= $u_1(t)$, voltage = straight] ++(0,-3) coordinate (B)
                (-0.5,0) to [inductor, n=l1] ++(0,-2) 
                to [short] ++(0,-1) coordinate (C)
                (0.5,0) to [inductor, n=l2, mirror] ++(0,-2) coordinate (D)
                (D) to [short, -o] (A)
                (C) to [short, -o] (B)
                (E) to [inductor, l_=$L_\mathrm{m}$, i=$i_\mathrm{m}(t)$, *-] ++(0,-2) 
                to [short, -*] ++(1.0,0);
                \path (l1.ul dot) node[circ]{}
                    (l2.ur dot) node[circ]{};
                \draw (l1.midtap) node[left]{$N_1$}
                (l2.midtap) node[right]{$N_2$};
                \draw[double, double distance=3pt, thick] let \p1=(l1.core west), \p2=(l2.core east) in (\x1/2+\x2/2, \y1) -- (\x1/2+\x2/2, \y2);
            \end{circuitikz}
            \caption{Switch-on time}
        \end{subfigure}
        \hspace{0.75cm}
        \onslide<2->{%
            \begin{subfigure}{0.45\textwidth}
                \centering
                \begin{circuitikz}[]
                    \draw (0.5,0) to [short] ++(0.5,0)
                    to [short, -o, i={$i_2(t)$}] ++(0.75,0)
                    to [open, o-o, v^= $u_2(t)$, voltage = straight] ++(0,-2) coordinate (A)
                    (-0.5,0) to ++(-1,0) coordinate (E)
                    to [short, -o, i_<={$i_1(t)=0$}] ++(-1.0,0)
                    to [open, o-o, v_= $u_1(t)$, voltage = straight] ++(0,-3) coordinate (B)
                    (-0.5,0) to [inductor, n=l1] ++(0,-2) 
                    to [open] ++(0,-1) coordinate (C)
                    (0.5,0) to [inductor, n=l2, mirror] ++(0,-2) coordinate (D)
                    (D) to [short, -o] (A)
                    (C) to [short, -o] (B)
                    (E) to [inductor, l_=$L_\mathrm{m}$, i=$i_\mathrm{m}(t)$, *-] ++(0,-2) 
                    to [short] ++(1.0,0);;
                    \path (l1.ul dot) node[circ]{}
                        (l2.ur dot) node[circ]{};
                    \draw (l1.midtap) node[left]{$N_1$}
                    (l2.midtap) node[right]{$N_2$};
                    \draw[double, double distance=3pt, thick] let \p1=(l1.core west), \p2=(l2.core east) in (\x1/2+\x2/2, \y1) -- (\x1/2+\x2/2, \y2);
                \end{circuitikz}
                \caption{Switch-off time}            
            \end{subfigure}
        }%
            \centering
        \caption{Switch states of the flyback converter}
        \label{fig:flyback-converter-switch-states}
    \end{figure}
\end{frame}

%%%%%%%%%%%%%%%%%%%%%%%%%%%%%%%%%%%%%%%%%%%%%%%%%%%%%%%%%%%%%
%% Flyback converter: steady-state time-domain behavior in CCM %%
%%%%%%%%%%%%%%%%%%%%%%%%%%%%%%%%%%%%%%%%%%%%%%%%%%%%%%%%%%%%%
\begin{frame}[fragile]
    \frametitle{Flyback converter: steady-state time-domain behavior in CCM}
    \begin{figure}
        \begin{tikzpicture}
            \pgfmathsetmacro{\D}{0.4} % duty cycle
            \pgfmathsetmacro{\n}{0.7} % turns ration n= N2/N1
            \pgfmathsetmacro{\gain}{\n*\D/(1-\D)} % current ripple
            \begin{groupplot}[group style={group size=1 by 3, xticklabels at = edge bottom}, height=0.34\textheight, width=0.875\textwidth, xmin=0, xmax=4, grid,clip = false, ymin = 0, ymax =1.1]

                % Top plot: voltage at the switch
                \nextgroupplot[ylabel = {$u_\mathrm{L_\mathrm{m}}(t)$}, ytick = {-1, -0.5, 0, 0.5, 1}, yticklabels = {$-U_1$, , 0, , $U_1$}, ymin = -1.1, ymax = 1.1]
                    \pgfplotsinvokeforeach{0,...,3}{
                        \edef\AddPlot{\noexpand\addplot[signalalpha, thick] coordinates {({0 + #1},-\gain) ({0 + #1},1) ({\D + #1},1) ({\D + #1},-\gain) ({1 + #1},-\gain) ({1 + #1},1)};}
                        \AddPlot
                    }
                    \node[above, inner sep = 2pt, anchor = south] at (axis cs:1.5+\D/2, -\gain) {$-U_2\frac{N_1}{N_2}$}; % label U_2
                    \draw [thick,<->]  (0,0.6) -- node[below]{$T_\mathrm{on}$}(\D, 0.6); % T_on 
                    \draw [thick,<->]  (\D,0.6) -- node[below]{$T_\mathrm{off}$}(1.0, 0.6); % T_off
                    \draw [thick,<->]  (0.0,-1.4) -- node[below]{$T_\mathrm{s}$}(1.0, -1.4); % T_s 


                % Middle plot: inductor current
                \nextgroupplot[ylabel = {$i_\mathrm{m}(t)$}, ytick = {0, 0.5, 1}, yticklabels = {0, $\overline{i}_\mathrm{m}$, }]
                    \pgfplotsinvokeforeach{0,...,3}{
                        \edef\AddPlot{\noexpand\addplot[signaldelta, thick] coordinates {({0 + #1},0.25) ({\D + #1},0.75) ({1 + #1},0.25)};}
                        \AddPlot
                    }
                    \draw[signaldelta, thick, dashed] (axis cs:0,0.5) -- (axis cs:4,0.5); % dashed line at average current
                    \draw[{Latex[length=2mm]}-, thin] (axis cs:\D+0.02,0.75) -- node[right=1mm, fill=white, inner sep = 1pt]{$\max\{i_\mathrm{m}\}$}(axis cs:\D+0.3,0.9); % indicate max current
                    \draw[-{Latex[length=2mm]}, thin] (axis cs:0.75,0.2) node[right=1mm, fill=white, inner sep = 1pt, anchor = east]{$\min\{i_\mathrm{m}\}$} -- (axis cs:1-0.02,0.25); % indicate min current
                    \draw[thin] (axis cs:2+\D/4,0.25+0.125) -- (axis cs:2+\D/4,0.75-0.125) -- (axis cs:2+\D*3/4,0.75-0.125); % indicate positive current slopde
                    \node[above, inner sep = 2pt, anchor = south, xshift = -4mm] at (axis cs:2+\D/2, 0.75-0.125) {$\nicefrac{U_1}{L_\mathrm{m}}$}; % label positive current slope
                    \draw[thin] (axis cs:2.25+3*\D/4,0.75-0.125) -- (axis cs:2.75+\D/4,0.75-0.125) -- (axis cs:2.75+\D/4,0.25+0.125); % indicate negative current slope
                    \node[above, inner sep = 2pt, anchor = south, xshift = 2mm] at (axis cs:2.5+\D/2, 0.75-0.125) {$\nicefrac{-U_2\frac{N_1}{N_2}}{L_\mathrm{m}}$}; % label negative current slope
                
                % Bottom plot: input current
                \nextgroupplot[ylabel = {$i_1(t)$}, xlabel={$t/T_\mathrm{s}$}, ytick = {0, 0.5, 1}, yticklabels = {0, ,}]
                    \pgfplotsinvokeforeach{0,...,3}{
                        \edef\AddPlot{\noexpand\addplot[signaldelta, thick] coordinates {({0 + #1},0.25) ({\D + #1},0.75) ({\D + #1},0) ({1 + #1},0) ({1 + #1},0.25)};}
                        \AddPlot
                    }
                    \draw[signaldelta, thick, dashed] (axis cs:0,0.5 * \D) -- (axis cs:4, 0.5 * \D); % dashed line at average current
                    \node[signaldelta, above, inner sep = 1pt, anchor = east, fill = white, xshift=-1mm] at (axis cs:\D, 0.5 * \D) {$\overline{i}_1$}; % label average current
                    \node[signaldelta, above, inner sep = 2pt, anchor = center, fill = white] at (axis cs:\D/2-0.05, 0.85) {$i_1(t)$}; % label input current
            \end{groupplot}

            % second y-axis for the bottom plot
					\begin{groupplot}[group style={group size=1 by 3, y descriptions at = edge right}, height=0.34\textheight, width=0.875\textwidth, xmin=0, xmax=4, grid,clip = false, xtick=\empty, axis line style=transparent, ymin = 0, ymax =1.1]
						\nextgroupplot[ytick = \empty]
							%
                        \nextgroupplot[ytick = \empty, ymin =-1.1]
                        %
						\nextgroupplot[ylabel = {$i_2(t)$}, ytick = {}, yticklabels = {}]
                            \pgfplotsinvokeforeach{0,...,3}{
                                \edef\AddPlot{\noexpand\addplot[signalalpha, thick] coordinates {({0 + #1},0) ({\D + #1},0) ({\D + #1},0.75/\n) ({1 + #1},0.25/\n) ({1 + #1},0)};}
                                \AddPlot
                            }
                            \node[signalalpha, above, inner sep = 2pt, anchor = center, fill = white] at (axis cs:0.95, 0.85) {$i_2(t)$}; % label output current
                            \draw[signalalpha, thick, dashed] (axis cs:0,0.5/\n-0.5*\D/\n) -- (axis cs:4, 0.5/\n-0.5*\D/\n); % dashed line at average current
                            \node[signalalpha, above, inner sep = 1pt, anchor = west, fill = white, xshift=1mm] at (axis cs:\D, 0.5/\n-0.5*\D/\n) {$\overline{i}_2$}; % label average current
					\end{groupplot}
        \end{tikzpicture}
    \end{figure}
\end{frame}

%%%%%%%%%%%%%%%%%%%%%%%%%%%%%%%%%%%%%%%%%%%%%%%%%%%%%%%%%%%%%
%% Flyback converter: impact of the transformer turns ratio %%
%%%%%%%%%%%%%%%%%%%%%%%%%%%%%%%%%%%%%%%%%%%%%%%%%%%%%%%%%%%%%
\begin{frame}
    \frametitle{Flyback converter: impact of the transformer turns ratio}
    \begin{columns}
        \begin{column}{0.5\textwidth}
           The transformer scales the peak input and output current according to the turns ratio $\nicefrac{N_2}{N_1}$ (with $\varepsilon$ being a small time period)
            \begin{equation*}
                i_2(t=T_\mathrm{on}+\varepsilon) = \frac{N_1}{N_2} i_1(t=T_\mathrm{on}-\varepsilon),
            \end{equation*}
            i.e., the output side may carry significantly different peak currents than the input.
            \onslide<2->{
            Also, when the transistor blocks it must withstand the voltage
            \begin{equation*}
                u_\mathrm{T}(t) = u_1 (t) + \frac{N_1}{N_2}u_2 (t), \quad t\in [T_\mathrm{on}, T_\mathrm{s}].
            \end{equation*}
            Hence, the turn ratio has a significant impact on components' stress factors.
            }%
        \end{column}
        %
        \begin{column}{0.5\textwidth}
            \begin{figure}
                \begin{tikzpicture}
                    \pgfmathsetmacro{\D}{0.6} % duty cycle
                    \pgfmathsetmacro{\n}{0.6} % turns ration n= N2/N1
                    \pgfmathsetmacro{\ioff}{0.2} % offset current / min current primary side
                    \begin{axis}[
                        xlabel={$t$},
                        ymin=0, ymax=1/\n+0.1,
                        xmin=-\D-0.1, xmax=(1-\D)+0.1,
                        width = \textwidth,
                        height = 0.7\textheight,
                        grid,
                        thick,
                        clip = true,
                        axis lines=middle,
                        ytick = {0, 1, 1/\n}, 
                        yticklabels = {0, $\max\{i_1\}$, $\frac{N_1}{N_2}\max\{i_1\}$},
                        xtick = {-\D, 0.001, 1-\D},
                        xticklabels = {$0$, $T_\mathrm{on}$, $T_\mathrm{s}$},
                        ]
                        \addplot[signaldelta] coordinates {(-\D,0) (-\D,\ioff) (0,1) (0,0) (1-\D, 0)};
                        \addplot[signalalpha] coordinates {(-\D,0)  (0,0) (0,1/\n) (1-\D, \ioff/\n) (1-\D, 0)};
                        \addplot[signalalpha, dashed] coordinates {(-1,1/\n) (-\D,\ioff/\n) (-\D,0)};
                        \addplot[signaldelta, dashed] coordinates {(1-\D, 0) (1-\D, \ioff) (1, 1)};
                        \node[signalalpha, above, inner sep = 2pt, anchor = south, fill = white] at (axis cs:\D/2, 1) {$i_2(t)$};
                        \node[signaldelta, above, inner sep = 2pt, anchor = center, fill = white] at (axis cs:-\D/2, 0.3) {$i_1(t)$};
                    \end{axis}
                \end{tikzpicture}
                \caption{Example of the ratio of the input and output current for $\nicefrac{N_2}{N_1}=0.6$}
            \end{figure}
        \end{column}
    \end{columns}
\end{frame}

%%%%%%%%%%%%%%%%%%%%%%%%%%%%%%%%%%%%%%%%%%%%%%%%%%%%%%%%%%%%%
%% Flyback converter: voltage voltage transfer ratio in CCM %%
%%%%%%%%%%%%%%%%%%%%%%%%%%%%%%%%%%%%%%%%%%%%%%%%%%%%%%%%%%%%%

\begin{frame}
    \frametitle{Flyback converter: voltage transfer ratio in CCM}
    \onslide<1->{
    In CCM, the voltage balance of the magnetizing inductor $L_\mathrm{m}$ delivers:
    \begin{equation}
        u_\mathrm{L_\mathrm{m}}(t) = \begin{cases} 
            U_1, & t\in [k T_\mathrm{s}, k T_\mathrm{s} + T_\mathrm{on}],\\ 
            -\frac{N_1}{N_2}U_2 & t\in [k T_\mathrm{s} + T_\mathrm{on}, (k+1) T_\mathrm{s}].
        \end{cases}
    \end{equation}}%
    \onslide<2->{In steady state, the average inductor voltage per period must be zero, yielding}%
    \begin{equation}
        \onslide<2->{U_1 T_\mathrm{on} = \frac{N_1}{N_2} U_2 T_\mathrm{off}  \quad}\onslide<3->{ \Leftrightarrow \quad  U_1 D T_\mathrm{s} = \frac{N_1}{N_2}U_2 (1-D) T_\mathrm{s} \quad}\onslide<4->{ \Leftrightarrow \quad \frac{U_2}{U_1} = \frac{N_2}{N_1}\frac{D}{1-D}.}
    \end{equation}
    \begin{itemize}
        \item<5-> Structurally similar result to the (inverting/synchronous) buck-boost converter.
        \item<6-> The voltage transfer ratio is additionally scaled by the turns ratio $\nicefrac{N_2}{N_1}$.
        \item<7-> The flyback's tranformer enables additional degrees of freedom to achieve a certain voltage transfer ratio via $D$ and $\nicefrac{N_2}{N_1}$.
    \end{itemize}
\end{frame}

%%%%%%%%%%%%%%%%%%%%%%%%%%%%%%%%%%%%%%%%%%%%%%%%%%%%%%%%%%%%%
%% Flyback converter: switch states %%
%%%%%%%%%%%%%%%%%%%%%%%%%%%%%%%%%%%%%%%%%%%%%%%%%%%%%%%%%%%%%

\begin{frame}
    \frametitle{Flyback converter: switch states in DCM}
    \onslide<1->{The flyback converter in DCM has three different switch states:}
    \begin{itemize}
            \item<1-> Transistor on-time:  $T_\mathrm{on}=DT_\mathrm{s}$,
            \item<2-> Transistor off-time (conducting diode): $T'_\mathrm{off}=D'T_\mathrm{s}$,
            \item<3-> Transistor off-time (no conduction):  $T''_\mathrm{off}=T_\mathrm{s}-T_\mathrm{on}-T'_\mathrm{off}$.
    \end{itemize}
    \begin{figure}
        \centering
        \onslide<1->{%	
        \begin{subfigure}{0.33\textwidth}
            \centering
            \begin{circuitikz}[scale=0.8, transform shape]
                \draw (0.5,0) to [short] ++(0.5,0)
                to [open]  ++(1.0,0)
                to [short, -o, i={$i_2=0$}] ++(0.75,0)
                to [open, o-o, v= $u_2$, voltage = straight] ++(0,-2) coordinate (A)
                (-0.5,0) to ++(-1,0) coordinate (E)
                to [short, -o, i_<=$i_1$] ++(-1.0,0)
                to [open, o-o, v^=$u_1$, voltage = straight] ++(0,-3) coordinate (B)
                (-0.5,0) to [inductor, n=l1] ++(0,-2) 
                to [short] ++(0,-1) coordinate (C)
                (0.5,0) to [inductor, n=l2, mirror] ++(0,-2) coordinate (D)
                (D) to [short, -o] (A)
                (C) to [short, -o] (B)
                (E) to [inductor, l_=$L_\mathrm{m}$, i=$i_\mathrm{m}$, *-] ++(0,-2) 
                to [short, -*] ++(1.0,0);
                \path (l1.ul dot) node[circ]{}
                    (l2.ur dot) node[circ]{};
                \draw (l1.midtap) node[left]{$N_1$}
                (l2.midtap) node[right]{$N_2$};
                \draw[double, double distance=3pt, thick] let \p1=(l1.core west), \p2=(l2.core east) in (\x1/2+\x2/2, \y1) -- (\x1/2+\x2/2, \y2);
            \end{circuitikz}
            \caption{Switch-on time $T_\mathrm{on}$}
        \end{subfigure}%
        }%
        \onslide<2->{%
        \begin{subfigure}{0.33\textwidth}
            \centering
            \hspace{-0.6cm}
            \begin{circuitikz}[scale=0.8, transform shape]
                \draw (0.5,0) to [short] ++(0.5,0)
                to [short, -o, i={$i_2$}] ++(0.75,0)
                to [open, o-o, v = $u_2$, voltage = straight] ++(0,-2) coordinate (A)
                (-0.5,0) to ++(-1,0) coordinate (E)
                to [short, -o, i_<={$i_1=0$}] ++(-1.0,0)
                to [open, o-o, v^= $u_1$, voltage = straight] ++(0,-3) coordinate (B)
                (-0.5,0) to [inductor, n=l1] ++(0,-2) 
                to [open] ++(0,-1) coordinate (C)
                (0.5,0) to [inductor, n=l2, mirror] ++(0,-2) coordinate (D)
                (D) to [short, -o] (A)
                (C) to [short, -o] (B)
                (E) to [inductor, l_=$L_\mathrm{m}$, i=$i_\mathrm{m}$, *-] ++(0,-2) 
                to [short] ++(1.0,0);;
                \path (l1.ul dot) node[circ]{}
                    (l2.ur dot) node[circ]{};
                \draw (l1.midtap) node[left]{$N_1$}
                (l2.midtap) node[right]{$N_2$};
                \draw[double, double distance=3pt, thick] let \p1=(l1.core west), \p2=(l2.core east) in (\x1/2+\x2/2, \y1) -- (\x1/2+\x2/2, \y2);
            \end{circuitikz}
            \caption{Switch-off time $T'_\mathrm{off}$}
        \end{subfigure}
        }%
        \onslide<3->{%
        \begin{subfigure}{0.33\textwidth}
            \centering
            \begin{circuitikz}[scale=0.8, transform shape]
                \draw (0.5,0) to [short] ++(0.5,0)
                to [open]  ++(1.0,0)
                to [short, -o, i={$i_2=0$}] ++(0.75,0)
                to [open, o-o, v= $u_2$, voltage = straight] ++(0,-2) coordinate (A)
                (-0.5,0) to ++(-1,0) coordinate (E)
                to [short, -o, i_<={$i_1=0$}] ++(-1.0,0)
                to [open, o-o, v^=$u_1$, voltage = straight] ++(0,-3) coordinate (B)
                (-0.5,0) to [inductor, n=l1] ++(0,-2) 
                to [open] ++(0,-1) coordinate (C)
                (0.5,0) to [inductor, n=l2, mirror] ++(0,-2) coordinate (D)
                (D) to [short, -o] (A)
                (C) to [short, -o] (B)
                (E) to [inductor, l_=$L_\mathrm{m}$, i=$i_\mathrm{m}$, *-] ++(0,-2) 
                to [short, -] ++(1.0,0);
                \path (l1.ul dot) node[circ]{}
                    (l2.ur dot) node[circ]{};
                \draw (l1.midtap) node[left]{$N_1$}
                (l2.midtap) node[right]{$N_2$};
                \draw[double, double distance=3pt, thick] let \p1=(l1.core west), \p2=(l2.core east) in (\x1/2+\x2/2, \y1) -- (\x1/2+\x2/2, \y2);
            \end{circuitikz}
            \caption{Switch-off time $T''_\mathrm{off}$}
        \end{subfigure}
        }%
    \caption{Switch states of the flyback converter including DCM} 
    \label{fig:flyback-switch-states-DCM}
    \end{figure}
\end{frame}

%%%%%%%%%%%%%%%%%%%%%%%%%%%%%%%%%%%%%%%%%%%%%%%%%%%%%%%%%%%%%
%% Flyback converter: steady-state time-domain behavior in DCM %%
%%%%%%%%%%%%%%%%%%%%%%%%%%%%%%%%%%%%%%%%%%%%%%%%%%%%%%%%%%%%%
\begin{frame}[fragile]
    \frametitle{Flyback converter: steady-state time-domain behavior in DCM}
    \begin{figure}
        \begin{tikzpicture}
            \pgfmathsetmacro{\D}{0.4} % duty cycle
            \pgfmathsetmacro{\Doff}{0.3} % relative off time (diode conducting)
            \pgfmathsetmacro{\n}{0.7} % turns ration n= N2/N1
            \pgfmathsetmacro{\gain}{\n*\D/(1-\D)} % current ripple
            \begin{groupplot}[group style={group size=1 by 3, xticklabels at = edge bottom}, height=0.34\textheight, width=0.875\textwidth, xmin=0, xmax=4, grid,clip = false, ymin = 0, ymax =1.1]

                % Top plot: voltage at the switch
                \nextgroupplot[ylabel = {$u_\mathrm{L_\mathrm{m}}(t)$}, ytick = {-1, -0.5, 0, 0.5, 1}, yticklabels = {$-U_1$, , 0, , $U_1$}, ymin = -1.1, ymax = 1.1]
                    \pgfplotsinvokeforeach{0,...,3}{
                        \edef\AddPlot{\noexpand\addplot[signalalpha, thick] coordinates {({0 + #1},0) ({0 + #1},1) ({\D + #1},1) ({\D + #1},-\gain) ({\D + \Doff + #1},-\gain) ({\D + \Doff + #1}, 0) ({1 + #1},0) ({1 + #1},1)};}
                        \AddPlot
                    }
                    \node[above, inner sep = 2pt, anchor = west] at (axis cs:0.85, -\gain) {$-U_2\frac{N_1}{N_2}$}; % label U_2
                    \draw [thick,<->]  (0,-1.4) -- node[below]{$T_\mathrm{on}$}(\D, -1.4); % T_on 
                    \draw [thick,<->]  (\D,-1.4) -- node[below]{$T'_\mathrm{off}$}(\D+\Doff, -1.4); % T'_off
                    \draw [thick,<->]  (\D+\Doff,-1.4) -- node[below]{$T''_\mathrm{off}$}(1, -1.4); % T''_off
                    \draw [thick,<->]  (1.0,-1.4) -- node[below]{$T_\mathrm{s}$}(2.0, -1.4); % T_s 


                % Middle plot: inductor current
                \nextgroupplot[ylabel = {$i_\mathrm{m}(t)$}, ytick = {0, 0.5, 1}, yticklabels = {0, , }]
                    \pgfplotsinvokeforeach{0,...,3}{
                        \edef\AddPlot{\noexpand\addplot[signaldelta, thick] coordinates {({0 + #1},0) ({\D + #1},0.75) ({\D + \Doff + #1},0) ({1 + #1},0)};}
                        \AddPlot
                    }
                    \draw[signaldelta, thick, dashed] (axis cs:0,0.75/2*\D+0.75/2*\Doff) -- (axis cs:4,0.75/2*\D+0.75/2*\Doff); % dashed line at average current
                    \draw[{Latex[length=2mm]}-, thin] (axis cs:\D+0.02,0.75) -- node[right=1mm, fill=white, inner sep = 1pt]{$\max\{i_\mathrm{m}\}$}(axis cs:\D+0.3,0.9); % indicate max current
                    \draw[thin] (axis cs:2+\D/4,0.75/4) -- (axis cs:2+\D/4,0.75/4*3) -- (axis cs:2+\D*3/4,0.75/4*3); % indicate positive current slopde
                    \node[above, inner sep = 2pt, anchor = south, xshift = -4mm] at (axis cs:2+\D/2, 0.75-0.165) {$\nicefrac{U_1}{L_\mathrm{m}}$}; % label positive current slope
                    \draw[thin] (axis cs:2+\D+\Doff/4,0.75*3/4) -- (axis cs:2+\D+\Doff*3/4,0.75*3/4) -- (axis cs:2+\D+\Doff*3/4,0.75/4); % indicate negative current slope
                    \node[above, inner sep = 2pt, anchor = south, xshift = 2mm] at (axis cs:2.5+\D/2, 0.75-0.175) {$\nicefrac{-U_2\frac{N_1}{N_2}}{L_\mathrm{m}}$}; % label negative current slope
                
                % Bottom plot: input current
                \nextgroupplot[ylabel = {$i_1(t)$}, xlabel={$t/T_\mathrm{s}$}, ytick = {0, 0.5, 1}, yticklabels = {0, ,}]
                    \pgfplotsinvokeforeach{0,...,3}{
                        \edef\AddPlot{\noexpand\addplot[signaldelta, thick] coordinates {({0 + #1},0) ({\D + #1},0.75) ({\D + #1},0) ({1 + #1},0) ({1 + #1},0)};}
                        \AddPlot
                    }
                    \draw[signaldelta, thick, dashed] (axis cs:0,0.75/2 * \D) -- (axis cs:4, 0.75/2 * \D); % dashed line at average current
                    \node[signaldelta, above, inner sep = 1pt, anchor = south, fill = white] at (axis cs:3-\Doff/2, 0.75/2 * \D) {$\overline{i}_1$}; % label average current
                    \node[signaldelta, above, inner sep = 2pt, anchor = center, fill = white] at (axis cs:\D/2-0.05, 0.8) {$i_1(t)$}; % label input current
            \end{groupplot}

            % second y-axis for the bottom plot
					\begin{groupplot}[group style={group size=1 by 3, y descriptions at = edge right}, height=0.34\textheight, width=0.875\textwidth, xmin=0, xmax=4, grid,clip = false, xtick=\empty, axis line style=transparent, ymin = 0, ymax =1.1]
						\nextgroupplot[ytick = \empty]
							%
                        \nextgroupplot[ytick = \empty, ymin =-1.1]
                        %
						\nextgroupplot[ylabel = {$i_2(t)$}, ytick = {}, yticklabels = {}]
                            \pgfplotsinvokeforeach{0,...,3}{
                                \edef\AddPlot{\noexpand\addplot[signalalpha, thick] coordinates {({0 + #1},0) ({\D + #1},0) ({\D + #1},0.75/\n) ({\D + \Doff + #1},0) ({1 + #1},0)};}
                                \AddPlot
                            }
                            \node[signalalpha, above, inner sep = 2pt, anchor = center, fill = white] at (axis cs:0.75, 0.75) {$i_2(t)$}; % label output current
                            \draw[signalalpha, thick, dashed] (axis cs:0,0.75/2*\Doff/\n) -- (axis cs:4, 0.75/2*\Doff/\n); % dashed line at average current
                            \node[signalalpha, above, inner sep = 1pt, anchor = south, fill = white] at (axis cs:2-\Doff/2, 0.75/2*\Doff/\n) {$\overline{i}_2$}; % label average current
					\end{groupplot}
        \end{tikzpicture}
    \end{figure}
\end{frame}

%%%%%%%%%%%%%%%%%%%%%%%%%%%%%%%%%%%%%%%%%%%%%%%%%%%%%%%%%%%%%
%% Flyback converter: DCM operation characteristics %%
%%%%%%%%%%%%%%%%%%%%%%%%%%%%%%%%%%%%%%%%%%%%%%%%%%%%%%%%%%%%%
\begin{frame}
    \frametitle{Flyback converter: DCM operation characteristics}
    \onslide<1->{In DCM operation 
    $$  \overline{i}_\mathrm{m} < \frac{\Delta i_\mathrm{\mathrm{m}}}{2} \quad \Rightarrow \quad U_2 \neq U_1 \frac{N_2}{N_1}\frac{D}{1-D}$$
    applies due to the non-conducting diode during  $T''_\mathrm{off}$.} \onslide<2->{To find the input-to-output voltage ratio in DCM, we again utilize the current ripple balance:}
    \begin{equation}
        \begin{alignedat}{2}
            \onslide<2->{\Delta i_\mathrm{m} &= \frac{U_1}{L_\mathrm{m}}T_\mathrm{on} = \frac{U_1}{L_\mathrm{m}}DT_\mathrm{s} \quad &&\mbox{(rising edge)},}\\
            \onslide<3->{\Delta i_\mathrm{\mathrm{m}} &= \frac{N_1}{N_2}\frac{U_2}{L_\mathrm{m}}T'_\mathrm{off} = \frac{N_1}{N_2}\frac{U_2}{L_\mathrm{m}}D'T_\mathrm{s} \quad &&\mbox{(falling edge)}.}
        \end{alignedat}
    \end{equation}
    \onslide<4->{Solving for $D'$ yields
    \begin{equation}
        D' = \frac{N_2}{N_1}\frac{U_1}{U_2}D.
    \end{equation}}
    \onslide<5->{The average load current is }
    \begin{equation}
        \onslide<5->{\overline{i}_2 = \frac{N_1}{N_2}\frac{\Delta i_\mathrm{m}}{2}\frac{T'_\mathrm{off}}{T_\mathrm{s}}} \onslide<6->{= \frac{N_1}{N_2}\frac{\Delta i_\mathrm{m,max}D}{2}D'}\onslide<7->{= \frac{N_1}{N_2}\frac{\Delta i_\mathrm{m,max}}{2}\frac{U_1}{U_2}D^2 \frac{N_2}{N_1}}\onslide<8->{= \frac{\Delta i_\mathrm{m,max}}{2}\frac{U_1}{U_2}D^2.}
            \label{eq:flyback-converter-average-output-current-DCM}
    \end{equation}
\end{frame}

%%%%%%%%%%%%%%%%%%%%%%%%%%%%%%%%%%%%%%%%%%%%%%%%%%%%%%%%%%%%%
%% Flyback converter: DCM operation characteristics (cont.)%%
%%%%%%%%%%%%%%%%%%%%%%%%%%%%%%%%%%%%%%%%%%%%%%%%%%%%%%%%%%%%%
\begin{frame}
    \frametitle{Flyback converter: DCM operation characteristics (cont.)}
    Solving \eqref{eq:flyback-converter-average-output-current-DCM} delivers the \hl{flyback converter voltage gain in DCM} as
    \begin{equation}
        \frac{U_2}{U_1} = \frac{D^2}{2} \frac{\Delta i_\mathrm{m,max}}{\overline{i}_2}.
        \label{eq:voltage-ratio-DCM-flyback}
    \end{equation}
    \pause%
    Since $\Delta i_\mathrm{m,max}$ also depends on $U_1$, the relation \eqref{eq:voltage-ratio-DCM-flyback} only holds for a given $U_1$. Hence, we can insert $\Delta i_\mathrm{m,max}=\nicefrac{T_\mathrm{s} \cdot U_1}{L}$ in \eqref{eq:flyback-converter-average-output-current-DCM} and solve for $U_2$ to receive
    \begin{equation}
        U_2= U_1^2 \frac{D^2}{2} \frac{T_\mathrm{s}}{L_\mathrm{m} \overline{i}_2}.
    \end{equation}
    \begin{itemize}
        \item<3-> Interestingly, the voltage gain in DCM seems independent of the turns ratio $\nicefrac{N_2}{N_1}$.
        \item<4-> Reason: output voltage $U_2$ depends on the (average) output current $\overline{i}_2$ which is inversely scaled by the turns ratio -- cf. cancelation of $\nicefrac{N_2}{N_1}$ in \eqref{eq:flyback-converter-average-output-current-DCM}.
        \item<5-> However, the transformer's magnetizing inductance is actually a function of the turns ratio $L_\mathrm{m}(N_1, N_2)$ (compare \href{https://github.com/IAS-Uni-Siegen/EMD_course}{Electrical Machines and Drives} course material). 
    \end{itemize}
\end{frame}

%%%%%%%%%%%%%%%%%%%%%%%%%%%%%%%%%%%%%%%%%%%%%%%%%%%%%%%%%%%%%
%% Outlook: multi-port (flyback) converter %%
%%%%%%%%%%%%%%%%%%%%%%%%%%%%%%%%%%%%%%%%%%%%%%%%%%%%%%%%%%%%%
\begin{frame}
    \frametitle{Outlook: multi-port (flyback) converter}
    \begin{figure}
        \begin{subfigure}{0.4\textwidth}
            \centering
            \begin{tikzpicture}
                \draw (-0.5,2) to[inductor, name=l1] ++(0,-2)
                (-0.5,2) to [short, -o, i<_=$i_1(t)$] ++(-1,0) coordinate (A1)
                (-0.5,0) to [short, -o] ++(-1,0) coordinate (A2)
                (A1) to [open, v=$u_1$, voltage = straight] (A2);
                \draw (0.5,2) to[inductor, name=l2, mirror] ++(0,-2)
                (0.5,2) to [short, -o, i=$i_2(t)$] ++(1,0) coordinate (B1)
                (0.5,0) to [short, -o] ++(1,0) coordinate (B2)
                (B1) to [open, v^=$u_2$, voltage = straight] (B2);
                \draw (0.5,-1) to[inductor, name=l3, mirror] ++(0,-2)
                (0.5,-1) to [short, -o, i=$i_3(t)$] ++(1,0) coordinate (C1)
                (0.5,-3) to [short, -o] ++(1,0) coordinate (C2)
                (C1) to [open, v^=$u_3$, voltage = straight] (C2);
                \draw[double, double distance=3pt, thick] let \p1=(l1.core west), \p2=(l3.core east) in (\x1/2+\x2/2, \y1) -- (\x1/2+\x2/2, \y2);
                \path (l1.ul dot) node[circ]{}
                (l2.ul dot) node[circ]{}
                (l3.ul dot) node[circ]{};
                \draw (l1.midtap) node[left]{$N_1$}
                (l2.midtap) node[right]{$N_2$}
                (l3.midtap) node[right]{$N_3$};
            \end{tikzpicture}
           \caption{Schematic symbol representation} 
        \end{subfigure}
        \begin{subfigure}{0.4\textwidth}
            \centering
            \begin{tikzpicture}
                \draw (-0.5,2) to[inductor, name=l1] ++(0,-2)
                (-0.5,2) to [short, -*, i<_=$i'_1(t)$] ++(-1,0) coordinate (A1)
                (-0.5,0) to [short, -*] ++(-1,0) coordinate (A2)
                (A1) to [inductor, l_=$L_\mathrm{m}$, i=$i_\mathrm{m}(t)$] (A2)
                (A1) to [short, -o, i<_=$i_1(t)$] ++(-1.5,0) coordinate (A11)
                (A2) to [short, -o] ++(-1.5,0) coordinate (A22)
                (A11) to [open, v=$u_1$, voltage = straight] (A22);
                \draw (0.5,2) to[inductor, name=l2, mirror] ++(0,-2)
                (0.5,2) to [short, -o, i=$i_2(t)$] ++(1,0) coordinate (B1)
                (0.5,0) to [short, -o] ++(1,0) coordinate (B2)
                (B1) to [open, v^=$u_2$, voltage = straight] (B2);
                \draw (0.5,-1) to[inductor, name=l3, mirror] ++(0,-2)
                (0.5,-1) to [short, -o, i=$i_3(t)$] ++(1,0) coordinate (C1)
                (0.5,-3) to [short, -o] ++(1,0) coordinate (C2)
                (C1) to [open, v^=$u_3$, voltage = straight] (C2);
                \draw[double, double distance=3pt, thick] let \p1=(l1.core west), \p2=(l3.core east) in (\x1/2+\x2/2, \y1) -- (\x1/2+\x2/2, \y2);
                \path (l1.ul dot) node[circ]{}
                (l2.ul dot) node[circ]{}
                (l3.ul dot) node[circ]{};
                \draw (l1.midtap) node[left]{$N_1$}
                (l2.midtap) node[right]{$N_2$}
                (l3.midtap) node[right]{$N_3$};
            \end{tikzpicture}
            \caption{Equivalent circuit model} 
        \end{subfigure}
        \caption{Multi-port (flyback) transformer: add multiple secondary windings to a common core to enable different input-to-output voltage ratios}
        \label{fig:transformer_model_multiport}
    \end{figure}
\end{frame}


%%%%%%%%%%%%%%%%%%%%%%%%%%%%%%%%%%%%%%%%%%%%%%%%%%%%%%%%%%%%%
%% Forward converter %%
%%%%%%%%%%%%%%%%%%%%%%%%%%%%%%%%%%%%%%%%%%%%%%%%%%%%%%%%%%%%%
\subsection{Forward converter}


%%%%%%%%%%%%%%%%%%%%%%%%%%%%%%%%%%%%%%%%%%%%%%%%%%%%%%%%%%%%%
%% Topology derivation based on the buck converter %%
%%%%%%%%%%%%%%%%%%%%%%%%%%%%%%%%%%%%%%%%%%%%%%%%%%%%%%%%%%%%%
\begin{frame}
    \frametitle{Topology derivation based on the buck converter}
    \begin{figure}
        \begin{circuitikz}[]
            \draw (0,0) to [short, i=$i_1(t)$] ++(0.75,0)
            to [short] ++(2.0,0) coordinate (A)
            to [inductor, l=$L$] ++(2.0,0)
            to [short, i=$i_2(t)$] ++(0.75,0)
            to [open, o-o, v^= $u_2(t)$, voltage = straight] ++(0,-2) coordinate (B)
            (0,0) to [open, o-o, v_= $u_1(t)$, voltage = straight] ++(0,-2.0) coordinate (D) 
            (A) to [diode, l=$D$, invert, *-*]  ++(0,-2) coordinate (C)
            (C) to [short, -o]  (B)
            (C) to [Tnpn, n=npn1, invert] (D);
            \draw let \p1 = (npn1.B) in node[anchor=south] at (\x1,\y1) {$T$};
            \draw [decorate,decoration={brace,amplitude=10pt,mirror,raise=0.5cm},yshift=0pt] (0,-2.0) -- (5.5,-2.0) node [black,midway,yshift=-0.6cm] {};
        \end{circuitikz}

        \begin{circuitikz}[]
            \draw (0,0) to [short, i=$i_1(t)$] ++(0.75,0)
            to [short] ++(2.0,0)
            to [inductor, n=l1] ++(0,-2) 
            to [Tnpn, n=npn1, invert] ++(-2.75,0) 
            (0,0) to [open, o-o, v_= $u_1(t)$, voltage = straight] ++(0,-2.0);
            \draw  (3.75,0) to [inductor, n=l2, mirror] ++(0,-2) 
            to [short] ++(2,0) coordinate (A)
            to [diode, l_=$D_2$, *-*, v^<= $u_\mathrm{s}$, voltage = straight] ++(0,2) coordinate (B)
            to [inductor, l=$L$] ++(2,0)
            to [short, i=$i_2(t)$] ++(0.75,0)
            to [open, o-o, v^= $u_2(t)$, voltage = straight] ++(0,-2)
            to [short] (A)
            (3.75,0) to [diode, l=$D_1$] (B);
            \draw (2,-0.25) to [open, v = $u_\mathrm{p}$, voltage = straight] ++(0,-1.5);
            \draw let \p1 = (npn1.B) in node[anchor=east] at (\x1,\y1) {$T$};
            \path (l1.ul dot) node[circ]{}
                  (l2.ul dot) node[circ]{};
            \draw (l1.midtap) node[left]{$N_1$}
            (l2.midtap) node[right]{$N_2$};
            \draw[double, double distance=3pt, thick, fill = shadecolor] let \p1=(l1.core west), \p2=(l2.core east) in (\x1/2+\x2/2, \y1) -- (\x1/2+\x2/2, \y2);
            % gray backgrounds
            \begin{scope}[on background layer]
                \node[rectangle, draw = shadecolor,	fill = shadecolor,	opacity=0.3, minimum width = 3cm, minimum height = 3.4cm] (B1) at (7.0,-1) {};
                \node[inner sep = 1pt, anchor = south, font=\small] at (B1.south) {Buck filter};
            \end{scope}
            \begin{scope}[on background layer]
                \node[rectangle, draw = shadecolor,	fill = shadecolor,	opacity=0.3, minimum width = 5.2cm, minimum height = 3.4cm] (B1) at (2.6,-1) {};
                \node[inner sep = 1pt, anchor = south, font=\small] at (B1.south) {Transformed input stage};
            \end{scope}
        \end{circuitikz}
    \end{figure}
\end{frame}

%%%%%%%%%%%%%%%%%%%%%%%%%%%%%%%%%%%%%%%%%%%%%%%%%%%%%%%%%%%%%
%% Forward converter: topology %%
%%%%%%%%%%%%%%%%%%%%%%%%%%%%%%%%%%%%%%%%%%%%%%%%%%%%%%%%%%%%%
\begin{frame}
    \frametitle{Forward converter: topology}
    \begin{columns}
        \begin{column}{0.4\textwidth}
            \begin{itemize}
                \item \hl{Forward converter = galvanically isolated buck converter.}
                \item<2-> Main energy buffer: inductor $L$.
                \item<3-> Transformer: galvanic isolation plus voltage scaling: $$u_\mathrm{s}(t)=\frac{N_2}{N_1}u_\mathrm{p}(t) $$ with $u_\mathrm{p}(t)=u_\mathrm{1}(t), t\in[0, T_\mathrm{on}]$.
                \item<4-> \hl{Different to flyback, where the transformer's purpose is to provide both  energy storage and galvanic isolation.}
            \end{itemize}
        \end{column}
        %
        \begin{column}{0.6\textwidth}
            \begin{figure}
                \begin{circuitikz}[]
                    \draw (0,0) to [short, i=$i_1(t)$] ++(1.5,0)
                    to [inductor, n=l1] ++(0,-2) 
                    to [Tnpn, n=npn1, invert] ++(0,-2)  -- (0,-4.0)
                    (0,0) to [open, o-o, v_= $u_1(t)$, voltage = straight] ++(0,-4.0);
                    \draw  (2.5,0) to [inductor, n=l2, mirror] ++(0,-2) 
                    to [short] ++(2,0) coordinate (A)
                    to [diode, l_=$D_2$, *-*, v^<= $u_\mathrm{s}$, voltage = straight] ++(0,2) coordinate (B)
                    to [inductor, l=$L$] ++(2,0)
                    to [short, i=$i_2(t)$] ++(0.75,0)
                    to [open, o-o, v^= $u_2(t)$, voltage = straight] ++(0,-2)
                    to [short] (A)
                    (2.5,0) to [diode, l=$D_1$] (B);
                    \draw (0.75,-0.25) to [open, v = $u_\mathrm{p}$, voltage = straight] ++(0,-1.5);
                    \draw let \p1 = (npn1.B) in node[anchor=east] at (\x1,\y1) {$T$};
                    \path (l1.ul dot) node[circ]{}
                          (l2.ul dot) node[circ]{};
                    \draw (l1.midtap) node[left]{$N_1$}
                    (l2.midtap) node[right]{$N_2$};
                    \draw[double, double distance=3pt, thick] let \p1=(l1.core west), \p2=(l2.core east) in (\x1/2+\x2/2, \y1) -- (\x1/2+\x2/2, \y2);
                \end{circuitikz}
                \caption{Forward converter topology}
                \label{fig:forward_converter_topology}
            \end{figure}
        \end{column}
    \end{columns}
\end{frame}

%%%%%%%%%%%%%%%%%%%%%%%%%%%%%%%%%%%%%%%%%%%%%%%%%%%%%%%%%%%%%
%% Forward converter: steady-state time-domain behavior (ideal transformer) %%
%%%%%%%%%%%%%%%%%%%%%%%%%%%%%%%%%%%%%%%%%%%%%%%%%%%%%%%%%%%%%
\begin{frame}[fragile]
    \frametitle{Forward converter: steady-state time-domain behavior (ideal transformer)}
    \begin{figure}
        \begin{tikzpicture}
            \pgfmathsetmacro{\D}{0.6} % duty cycle
            \begin{groupplot}[group style={group size=1 by 3, xticklabels at = edge bottom}, height=0.34\textheight, width=0.875\textwidth, xmin=0, xmax=4, grid,clip = false, ymin = 0, ymax =1.1]

                % Top plot: voltage at the switch
                \nextgroupplot[ylabel = {$u_\mathrm{s}(t)$}, ytick = {0, 0.5, 1}, yticklabels = {0, , $\frac{N_2}{N_1}U_1$}]
                    \pgfplotsinvokeforeach{0,...,3}{
                        \edef\AddPlot{\noexpand\addplot[signalalpha, thick] coordinates {({0 + #1},0) ({0 + #1},1) ({\D + #1},1) ({\D + #1},0) ({1 + #1},0) ({1 + #1},1)};}
                        \AddPlot
                    }
                    \draw[signalalpha, thick, dashed] (axis cs:0, \D) -- (axis cs:4, \D); % dashed line at U_2 (average)
                    \node[above, inner sep = 2pt, anchor = south] at (axis cs:1.5+\D/2, \D) {$U_2$}; % label U_2
                    \draw [thick,<->]  (0,0.5) -- node[below]{$T_\mathrm{on}$}(\D, 0.5); % T_on 
                    \draw [thick,<->]  (\D,0.5) -- node[below]{$T_\mathrm{off}$}(1.0, 0.5); % T_off
                    \draw [thick,<->]  (0.0,-0.2) -- node[below]{$T_\mathrm{s}$}(1.0, -0.2); % T_s 


                % Middle plot: inductor current
                \nextgroupplot[ylabel = {$i_\mathrm{L}(t)$}, ytick = {0, 0.5, 1}, yticklabels = {0, $\overline{i}_\mathrm{L}$, }]
                    \pgfplotsinvokeforeach{0,...,3}{
                        \edef\AddPlot{\noexpand\addplot[signaldelta, thick] coordinates {({0 + #1},0.25) ({\D + #1},0.75) ({1 + #1},0.25)};}
                        \AddPlot
                    }
                    \draw[signaldelta, thick, dashed] (axis cs:0,0.5) -- (axis cs:4,0.5); % dashed line at average current
                    \draw[{Latex[length=2mm]}-, thin] (axis cs:\D+0.02,0.75) -- node[right=1mm, fill=white, inner sep = 1pt]{$\max\{i_\mathrm{L}\}$}(axis cs:\D+0.3,0.9); % indicate max current
                    \draw[-{Latex[length=2mm]}, thin] (axis cs:0.75,0.2) node[right=1mm, fill=white, inner sep = 1pt, anchor = east]{$\min\{i_\mathrm{L}\}$} -- (axis cs:1-0.02,0.25); % indicate min current
                    \draw[thin] (axis cs:2+\D/4,0.25+0.125) -- (axis cs:2+\D/4,0.75-0.125) -- (axis cs:2+\D*3/4,0.75-0.125); % indicate positive current slopde
                    \node[above, inner sep = 2pt, anchor = south, xshift = -4mm] at (axis cs:2+\D/2, 0.75-0.125) {$\nicefrac{(\frac{N_2}{N_1}U_1-U_2)}{L}$}; % label positive current slope
                    \draw[thin] (axis cs:2.25+3*\D/4,0.75-0.125) -- (axis cs:2.75+\D/4,0.75-0.125) -- (axis cs:2.75+\D/4,0.25+0.125); % indicate negative current slope
                    \node[above, inner sep = 2pt, anchor = south, xshift = 0mm] at (axis cs:2.5+\D/2, 0.75-0.125) {$\nicefrac{-U_2}{L}$}; % label negative current slope
                
                % Bottom plot: input current
                \nextgroupplot[ylabel = {$i_1(t)$}, xlabel={$t/T_\mathrm{s}$}, ytick = {0, 0.5, 1}, yticklabels = {0, ,}]
                    \pgfplotsinvokeforeach{0,...,3}{
                        \edef\AddPlot{\noexpand\addplot[signaldelta, thick] coordinates {({0 + #1},0.25) ({\D + #1},0.75) ({\D + #1},0) ({1 + #1},0) ({1 + #1},0.25)};}
                        \AddPlot
                    }
                    \draw[signaldelta, thick, dashed] (axis cs:0,0.5 * \D) -- (axis cs:4, 0.5 * \D); % dashed line at average current
                    \node[above, inner sep = 2pt, anchor = south, fill = white] at (axis cs:1.5+\D/2, 0.5 * \D) {$\overline{i}_1$}; % label average current
            \end{groupplot}
        \end{tikzpicture}
    \end{figure}
\end{frame}

%%%%%%%%%%%%%%%%%%%%%%%%%%%%%%%%%%%%%%%%%%%%%%%%%%%%%%%%%%%%%
%% Forward converter: idealized steady-state operation %%
%%%%%%%%%%%%%%%%%%%%%%%%%%%%%%%%%%%%%%%%%%%%%%%%%%%%%%%%%%%%%
\begin{frame}
    \frametitle{Forward converter: idealized steady-state operation}
    Assumption:
    \begin{itemize}
        \item The transformer is ideal and does not exhibit a magnetizing inductance.
    \end{itemize}\pause
    Consequence:
    \begin{itemize}
        \item The transformer's secondary output voltage $u_\mathrm{s}(t)$ is a $\nicefrac{N_2}{N_1}$ scaled version of the standard buck converter's switch voltage (compare \figref{fig:step-down-converter-realization-1Q}). \pause
        \item The (idealized) forward converter characteristics are analogous to the buck converter.
    \end{itemize} \pause
    Hence, the \hl{voltage input-to-output voltage ratios for the (idealized) forward converter} are:
    \begin{equation}
        \mbox{CCM:}\quad \frac{U_2}{U_1} = \frac{N_2}{N_1}D, \qquad \mbox{DCM:}\quad U_2 = \frac{N_2^2}{N_1^2}\frac{D^2T_\mathrm{s}U_1^2}{D^2T_\mathrm{s}\frac{N_2}{N_1}U_1+2L\overline{i}_2}.
    \end{equation}
\end{frame}

%%%%%%%%%%%%%%%%%%%%%%%%%%%%%%%%%%%%%%%%%%%%%%%%%%%%%%%%%%%%%
%% Forward converter: magnetizing inductance issue %%
%%%%%%%%%%%%%%%%%%%%%%%%%%%%%%%%%%%%%%%%%%%%%%%%%%%%%%%%%%%%%
\begin{frame}
    \frametitle{Forward converter: magnetizing inductance issue}
    \begin{columns}
        \begin{column}{0.4\textwidth}
            \begin{varblock}{Magnetizing inductance}
                    With every switching cycle the primary magnetizing current $i_\mathrm{\mathrm{m}}(t)$ increases (i.e., transformer saturates and takes damage).
            \end{varblock}
            \begin{figure}
                \begin{tikzpicture}
                    \pgfmathsetmacro{\D}{0.6} % duty cycle
                    \begin{axis}[
                        xlabel={$t/T_\mathrm{s}$},
                        ymin=0, ymax=2.4,
                        xmin=0, xmax=2.25,
                        width = \textwidth,
                        height = 0.5\textheight,
                        grid,
                        thick,
                        clip = true,
                        ytick = {0, 1, 2}, 
                        yticklabels = {$0$, $1\frac{U_1}{L_\mathrm{m}}T_\mathrm{on}$, $2 \frac{U_1}{L_\mathrm{m}}T_\mathrm{on}$}
                        ]
                        \addplot[signaldelta] coordinates {(0,0) (\D,1) (1,1) (1+\D, 2) (2,2)};
                        \addplot[signaldelta, dashed] coordinates {(2,2) (2+\D,3)};
                        \node[signaldelta, above, inner sep = 2pt, anchor = south, fill=white] at (axis cs:0.6, 1.1) {$i_\mathrm{m}(t)$};
                    \end{axis}
                \end{tikzpicture}
            \end{figure}
        \end{column}
        %
        \begin{column}{0.6\textwidth}
            \begin{figure}
                \begin{circuitikz}[]
                    \draw (0,0) to [short, i=$i_1(t)$] ++(0.75,0) coordinate (A1)
                    to [short] ++(1,0)
                    to [inductor, n=l1] ++(0,-2) coordinate (A2)
                    to [Tnpn, n=npn1, invert] ++(0,-2)   -- (0,-4.0)
                    (0,0) to [open, o-o, v_= $u_1(t)$, voltage = straight] ++(0,-4.0);
                    \draw  (2.5,0) to [inductor, n=l2, mirror] ++(0,-2) 
                    to [short] ++(2,0) coordinate (A)
                    to [diode, l_=$D_2$, *-*] ++(0,2) coordinate (B)
                    to [inductor, l=$L$] ++(2,0)
                    to [short, i=$i_2(t)$] ++(0.75,0)
                    to [open, o-o, v^= $u_2(t)$, voltage = straight] ++(0,-2)
                    to [short] (A)
                    (2.5,0) to [diode, l=$D_1$] (B)
                    (A1) to [inductor, l_=$L_\mathrm{m}$, i=$i_\mathrm{m}(t)$, *-] ++(-0,-2)
                    to [short, -*] (A2);
                    \draw let \p1 = (npn1.B) in node[anchor=east] at (\x1,\y1) {$T$};
                    \path (l1.ul dot) node[circ]{}
                          (l2.ul dot) node[circ]{};
                    \draw (l1.midtap) node[left]{$N_1$}
                    (l2.midtap) node[right]{$N_2$};
                    \draw[double, double distance=3pt, thick] let \p1=(l1.core west), \p2=(l2.core east) in (\x1/2+\x2/2, \y1) -- (\x1/2+\x2/2, \y2);
                \end{circuitikz}
                \caption{Forward converter topology with primary magnetizing inductance}
                \label{fig:forward_converter_topology_magnetizing_inductance}
            \end{figure}
        \end{column}
    \end{columns}
\end{frame}

%%%%%%%%%%%%%%%%%%%%%%%%%%%%%%%%%%%%%%%%%%%%%%%%%%%%%%%%%%%%%
%% Forward converter: demagnetization via negative input voltage %%
%%%%%%%%%%%%%%%%%%%%%%%%%%%%%%%%%%%%%%%%%%%%%%%%%%%%%%%%%%%%%
\begin{frame}
    \frametitle{Forward converter: demagnetization via negative input voltage}
    \vspace{-0.3cm}
        \begin{figure}
            \begin{circuitikz}[]
                %Asym. half-bridge
                \draw (0,4) coordinate (A) to [open, o-o, v = $u_1(t)$, voltage = straight] ++(0,-6) coordinate (B)
                (A) to [short, o-, i=$i_1(t)$] ++(2,0) coordinate (E)
                to [Tnpn, n=npn1, invert] ++(0,-2) coordinate (C)
                to [short,*-] ++(2,0)  
                to [short] ++(1,0) coordinate (J)
                to [short] ++(1,0) coordinate (G)
                (C) to [short] ++(0,-2) 
                to [diode, l_=$D_1$, invert] ++(0,-2) coordinate (D)
                (E) to [short, *-] ++(2,0)
                to [diode, l_=$D_2$, invert] ++(0,-2)
                to [short] ++(0,-2) coordinate (F)
                to [Tnpn, n=npn2, invert] ++(0,-2) 
                to [short, -*] ++(-2,0)
                to [short, -o] (B)
                (F) to [short,*-] ++(1,0) coordinate (I)
                to [short] ++(1,0) coordinate (H)
                (J) to [open,v_= $u_\mathrm{p}(t)$, voltage = straight] (I);
                \draw let \p1 = (npn1.B) in node[anchor=east] at (\x1,\y1) {$T_1$};
                \draw let \p1 = (npn2.B) in node[anchor=east] at (\x1,\y1) {$T_2$};


                % Transformer + buck filter
                \draw (G) to [short] ++(1,0) coordinate (A1)
                to [inductor, n=l1] ++(0,-2) coordinate (B1)
                (A1) to [open] ++(0.75,0) to [inductor, n=l2, mirror] ++(0,-2) 
                to [short] ++(2,0) coordinate (C1)
                to [diode, l_=$D_4$, *-*, v^<= $u_\mathrm{s}(t)$, voltage = straight] ++(0,2) coordinate (D1)
                to [inductor, l=$L$] ++(2,0)
                to [short, i=$i_2(t)$] ++(0.75,0)
                to [open, o-o, v^= $u_2(t)$, voltage = straight] ++(0,-2)
                to [short] (C1)
                (A1) to [open] ++(0.75,0) to [diode, l=$D_3$] (D1)
                (G) to [inductor, l_=$L_\mathrm{m}$, i=$i_\mathrm{m}(t)$, *-*] ++(-0,-2)
                to [short] (B1);
                \path (l1.ul dot) node[circ]{}
                        (l2.ul dot) node[circ]{};
                \draw (l1.midtap) node[left]{$N_1$}
                (l2.midtap) node[right]{$N_2$};
                \draw[double, double distance=3pt, thick] let \p1=(l1.core west), \p2=(l2.core east) in (\x1/2+\x2/2, \y1) -- (\x1/2+\x2/2, \y2);
            \end{circuitikz}
            \caption{Forward converter topology with an asymmetrical half-bridge}
            \label{fig:forward_converter_topology_asymmetrical_half_bridge}
        \end{figure}
\end{frame}

%%%%%%%%%%%%%%%%%%%%%%%%%%%%%%%%%%%%%%%%%%%%%%%%%%%%%%%%%%%%%
%% Forward converter: steady-state time-domain behavior (asym. half-bridge) %%
%%%%%%%%%%%%%%%%%%%%%%%%%%%%%%%%%%%%%%%%%%%%%%%%%%%%%%%%%%%%%
\begin{frame}[fragile]
    \frametitle{Forward converter: steady-state time-domain behavior (asym. half-bridge)}
    \begin{figure}
        \begin{tikzpicture}
            \pgfmathsetmacro{\D}{0.35} % duty cycle
            \begin{groupplot}[group style={group size=1 by 3, xticklabels at = edge bottom}, height=0.34\textheight, width=0.875\textwidth, xmin=0, xmax=4, grid,clip = false, ymin = 0, ymax =1.1]

                % Top plot: input voltage a primary side
                \nextgroupplot[ylabel = {$u_\mathrm{p}(t)$}, ytick = {-1, 0, 1}, yticklabels = {$-U_1$, 0, $U_1$}, ymin = -1.1, ymax =1.1]
                    \pgfplotsinvokeforeach{0,...,3}{
                        \edef\AddPlot{\noexpand\addplot[signalalpha, thick] coordinates {({0 + #1},0) ({0 + #1},1) ({\D + #1},1) ({\D + #1},-1) ({2*\D + #1},-1) ({2*\D + #1},0) ({1 + #1},0)};}
                        \AddPlot
                    }
                    \draw [thick,<->]  (0,0) -- node[below]{$D T_\mathrm{s}$}(\D, 0);  
                    \draw [thick,<->]  (\D,0) -- node[below]{$D T_\mathrm{s}$}(2*\D, 0); 
                    \draw [thick,<->]  (0.0,-1.45) -- node[below]{$T_\mathrm{s}$}(1.0, -1.45);  


                % Middle plot: switch voltage at buck side
                \nextgroupplot[ylabel = {$u_\mathrm{s}(t)$}, ytick = {0, 0.5, 1}, yticklabels = {0, , $\frac{N_2}{N_1}U_1$}]
                    \pgfplotsinvokeforeach{0,...,3}{
                        \edef\AddPlot{\noexpand\addplot[signalalpha, thick] coordinates {({0 + #1},0) ({0 + #1},1) ({\D + #1},1) ({\D + #1},0) ({1 + #1},0) ({1 + #1},1)};}
                        \AddPlot
                    }
                    \draw[signalalpha, thick, dashed] (axis cs:0, \D) -- (axis cs:4, \D); % dashed line at U_2 (average)
                    \node[above, inner sep = 2pt, anchor = south] at (axis cs:1.5+\D/2, \D) {$U_2$}; % label U_2
                    \draw [thick,<->]  (0.0,-0.2) -- node[below]{$T_\mathrm{on}$}(\D, -0.2);  
                    \draw [thick,<->]  (\D,-0.2) -- node[below]{$T_\mathrm{off}$}(1, -0.2);  
    
                
                % Bottom plot: magnetizing current
                \nextgroupplot[ylabel = {$i_\mathrm{m}(t)$}, xlabel={$t/T_\mathrm{s}$}, ytick = {0, 0.5, 1}, yticklabels = {0, ,}]
                    \pgfplotsinvokeforeach{0,...,3}{
                        \edef\AddPlot{\noexpand\addplot[signaldelta, thick] coordinates {({0 + #1},0) ({\D + #1},0.6) ({2*\D + #1},0) ({1 + #1},0)};}
                        \AddPlot
                    }
            \end{groupplot}
        \end{tikzpicture}
    \end{figure}
\end{frame}

%%%%%%%%%%%%%%%%%%%%%%%%%%%%%%%%%%%%%%%%%%%%%%%%%%%%%%%%%%%%%
%% Forward converter with asym. half-bridge input stage %%
%%%%%%%%%%%%%%%%%%%%%%%%%%%%%%%%%%%%%%%%%%%%%%%%%%%%%%%%%%%%%
\begin{frame}
    \frametitle{Forward converter with asym. half-bridge input stage}
    To demagnetize the transformer, the input voltage $u_\mathrm{p}(t)$ is modulated as follows:
    \begin{equation}
        u_\mathrm{p}(t) = \begin{cases}
            U_1, &  t\in[kT_\mathrm{s}, kT_\mathrm{s}+D T_\mathrm{s}], \quad T_1=T_2=\mathrm{on},\\
            -U_1, & t\in[kT_\mathrm{s}+D T_\mathrm{s}, kT_\mathrm{s}+2D T_\mathrm{s}], \quad  T_1=T_2=\mathrm{off},\\
            0, &  t\in[kT_\mathrm{s}+2D T_\mathrm{s}, kT_\mathrm{s}+T_\mathrm{s}], \quad T_1=\mathrm{on},\, T_2=\mathrm{off}.
        \end{cases}  
    \end{equation}\pause
    Consequently, we have
    \begin{equation}
        \overline{u}_\mathrm{L_\mathrm{m}} = \frac{1}{T_\mathrm{s}}\int_{0}^{T_\mathrm{s}}u_\mathrm{p}(t)\mathrm{d}t = 0
    \label{eq:average_magnetizing_voltage_asym_half-bridge_forward_converter}
    \end{equation}
    and, therefore, the transformer's magnetizing current $i_\mathrm{m}(t)$ does not increase during a pulse period.\pause However, this also \hl{limits the applicable duty cycle} to 
    $$
    D\leq\frac{1}{2}
    $$
    since otherwise \eqref{eq:average_magnetizing_voltage_asym_half-bridge_forward_converter} cannot be fulfilled.
\end{frame}

%%%%%%%%%%%%%%%%%%%%%%%%%%%%%%%%%%%%%%%%%%%%%%%%%%%%%%%%%%%%%
%% Forward converter: demagnetization via negative input voltage (cont.) %%
%%%%%%%%%%%%%%%%%%%%%%%%%%%%%%%%%%%%%%%%%%%%%%%%%%%%%%%%%%%%%
\begin{frame}
    \frametitle{Forward converter: demagnetization via negative input voltage (cont.)}
    \vspace{-0.3cm}
        \begin{figure}
            \begin{circuitikz}[]
                %full-bridge
                \draw (0,4) coordinate (A) to [open, o-o, v = $u_1(t)$, voltage = straight] ++(0,-6) coordinate (B)
                (A) to [short, o-, i=$i_1(t)$] ++(2,0) coordinate (E)
                to [Tnpn, n=npn1, invert, bodydiode] ++(0,-2) coordinate (C)
                to [short,*-] ++(2,0)  
                to [short] ++(1,0) coordinate (J)
                to [short] ++(1,0) coordinate (G)
                (C) to [short] ++(0,-2) 
                to [Tnpn, n=npn2, invert, bodydiode] ++(0,-2) coordinate (D)
                (E) to [short, *-] ++(2,0)
                to [Tnpn, n=npn3, invert, bodydiode] ++(0,-2)
                to [short] ++(0,-2) coordinate (F)
                to [Tnpn, n=npn4, invert, bodydiode] ++(0,-2) 
                to [short, -*] ++(-2,0)
                to [short, -o] (B)
                (F) to [short,*-] ++(1,0) coordinate (I)
                to [short] ++(1,0) coordinate (H)
                (J) to [open,v_= $u_\mathrm{p}(t)$, voltage = straight] (I);
                \draw let \p1 = (npn1.B) in node[anchor=east] at (\x1,\y1) {$T_1$};
                \draw let \p1 = (npn2.B) in node[anchor=east] at (\x1,\y1) {$T_2$};
                \draw let \p1 = (npn3.B) in node[anchor=east] at (\x1,\y1) {$T_3$};
                \draw let \p1 = (npn4.B) in node[anchor=east] at (\x1,\y1) {$T_4$};
                

                % Transformer + buck filter
                \draw (G) to [short] ++(1,0) coordinate (A1)
                to [inductor, n=l1] ++(0,-2) coordinate (B1)
                (A1) to [open] ++(0.75,0) to [inductor, n=l2, mirror] ++(0,-2) 
                to [short] ++(2,0) coordinate (C1)
                to [diode, l_=$D_2$, *-*, v^<= $u_\mathrm{s}(t)$, voltage = straight] ++(0,2) coordinate (D1)
                to [inductor, l=$L$] ++(2,0)
                to [short, i=$i_2(t)$] ++(0.75,0)
                to [open, o-o, v^= $u_2(t)$, voltage = straight] ++(0,-2)
                to [short] (C1)
                (A1) to [open] ++(0.75,0) to [diode, l=$D_1$] (D1)
                (G) to [inductor, l_=$L_\mathrm{m}$, i=$i_\mathrm{m}(t)$, *-*] ++(-0,-2)
                to [short] (B1);
                \path (l1.ul dot) node[circ]{}
                        (l2.ul dot) node[circ]{};
                \draw (l1.midtap) node[left]{$N_1$}
                (l2.midtap) node[right]{$N_2$};
                \draw[double, double distance=3pt, thick] let \p1=(l1.core west), \p2=(l2.core east) in (\x1/2+\x2/2, \y1) -- (\x1/2+\x2/2, \y2);
            \end{circuitikz}
            \caption{Forward converter topology with a full-bridge}
            \label{fig:forward_converter_topology_asymmetrical_full_bridge}
        \end{figure}
\end{frame}


%%%%%%%%%%%%%%%%%%%%%%%%%%%%%%%%%%%%%%%%%%%%%%%%%%%%%%%%%%%%%
%% Forward converter: steady-state time-domain behavior (full-bridge) %%
%%%%%%%%%%%%%%%%%%%%%%%%%%%%%%%%%%%%%%%%%%%%%%%%%%%%%%%%%%%%%
\begin{frame}[fragile]
    \frametitle{Forward converter: steady-state time-domain behavior (full-bridge)}
    \begin{figure}
        \begin{tikzpicture}
            \pgfmathsetmacro{\D}{0.35} % duty cycle
            \begin{groupplot}[group style={group size=1 by 3, xticklabels at = edge bottom}, height=0.34\textheight, width=0.875\textwidth, xmin=0, xmax=4, grid,clip = false, ymin = 0, ymax =1.1]

                % Top plot: input voltage a primary side
                \nextgroupplot[ylabel = {$u_\mathrm{p}(t)$}, ytick = {-1, 0, 1}, yticklabels = {$-U_1$, 0, $U_1$}, ymin = -1.1, ymax =1.1]
                    \pgfplotsinvokeforeach{0,...,3}{
                        \edef\AddPlot{\noexpand\addplot[signalalpha, thick] coordinates {({0 + #1},0) ({0 + #1},1) ({\D + #1},1) ({\D + #1},0) ({1/2 + #1},0) ({1/2 + #1},-1) ({1/2+\D + #1},-1) ({1/2+\D + #1},0) ({1 + #1},0)};}
                        \AddPlot
                    }
                    \draw [thick,<->]  (0,0) -- node[below]{$D T_\mathrm{s}$}(\D, 0);  
                    \draw [thick,<->]  (1/2,0) -- node[above]{$D T_\mathrm{s}$}(1/2+\D, 0); 
                    \draw [thick,<->]  (0.0,-1.45) -- node[below]{$T_\mathrm{s}$}(1.0, -1.45);  


                % Middle plot: switch voltage at buck side
                \nextgroupplot[ylabel = {$u_\mathrm{s}(t)$}, ytick = {0, 0.5, 1}, yticklabels = {0, , $\frac{N_2}{N_1}U_1$}]
                    \pgfplotsinvokeforeach{0,...,3}{
                        \edef\AddPlot{\noexpand\addplot[signalalpha, thick] coordinates {({0 + #1},0) ({0 + #1},1) ({\D + #1},1) ({\D + #1},0) ({1 + #1},0) ({1 + #1},1)};}
                        \AddPlot
                    }
                    \draw[signalalpha, thick, dashed] (axis cs:0, \D) -- (axis cs:4, \D); % dashed line at U_2 (average)
                    \node[above, inner sep = 2pt, anchor = south] at (axis cs:1.5+\D/2, \D) {$U_2$}; % label U_2
                    \draw [thick,<->]  (0.0,-0.2) -- node[below]{$T_\mathrm{on}$}(\D, -0.2);  
                    \draw [thick,<->]  (\D,-0.2) -- node[below]{$T_\mathrm{off}$}(1, -0.2);  
    
                
                % Bottom plot: magnetizing current
                \nextgroupplot[ylabel = {$i_\mathrm{m}(t)$}, xlabel={$t/T_\mathrm{s}$}, ytick = {-0.5, 0, 0.5}, yticklabels = {, 0,}, ymin = -0.6, ymax =0.6]
                    \pgfplotsinvokeforeach{0,...,3}{
                        \edef\AddPlot{\noexpand\addplot[signaldelta, thick] coordinates {({0 + #1},-0.3) ({\D + #1},0.3) ({1/2 + #1},0.3) ({1/2+\D + #1},-0.3) ({1 + #1},-0.3)};}
                        \AddPlot
                    }
            \end{groupplot}
        \end{tikzpicture}
    \end{figure}
\end{frame}

%%%%%%%%%%%%%%%%%%%%%%%%%%%%%%%%%%%%%%%%%%%%%%%%%%%%%%%%%%%%%
%% Forward converter: hysteresis curves of the transformer %%
%%%%%%%%%%%%%%%%%%%%%%%%%%%%%%%%%%%%%%%%%%%%%%%%%%%%%%%%%%%%%
\begin{frame}
    \frametitle{Forward converter: hysteresis curves of the transformer}
    \begin{figure}
        \centering
        \begin{subfigure}{0.45\textwidth}
            \centering
            \begin{tikzpicture}
                \tikzmath{
                    real \a, \b, \c, \d, \hn, \hc, \hm, \bc, \hcc;
                    \a = 6.0;
                    \b = 1.7;
                    \c = 1.5;
                    \d = 3.0;
                    \hn = 1.7; % nominal, utilized H field value
                    \hm = 7.0; % maximal H field value
                    \hc = -\hn + 2*\c/\b; % hyteresis return H value for nominal H field
                    \bc = \a/(1 + exp(-\b*\hc+\c))-\d; % hyteresis return B value for nominal H field
                    \hcc = (-ln(\a/\d-1)+\c)/\b; % coercive H field value
                }
                \begin{axis}[very thick,
                             samples = 100,
                             xlabel = $H$,
                             ylabel = $B$,
                             xmin = -\hm,
                             xmax = \hm,
                             ymin = -4,
                             ymax = 4,
                             axis x line = middle,
                             axis y line = middle,
                             ticks = none,
                             width=\textwidth]
                    \addplot[shadecolor, dashed, domain = -\hm:\hm] {\a/(1 + exp(-\b*x+\c))-\d};
                    \addplot[shadecolor, dashed ,domain = -\hm:\hm] {\a/(1 + exp(-\b*x-\c))-\d};
                    \addplot[signaldelta, name path=A, domain=\hcc:\hn, samples = 100] {\a/(1 + exp(-\b*x+\c))-\d};
                    \addplot[signaldelta, name path=B, domain=-\hcc:-\hc, samples = 100] {\a/(1 + exp(-\b*x-\c))-\d};
                    \addplot[shadecolor, opacity=0.3] fill between[of=A and B];
                    \addplot[signaldelta] coordinates {(-\hc, -\bc) (\hn, -\bc)};
                    \addplot[signaldelta] coordinates {(-\hcc, 0) (\hcc, 0)};
                \end{axis}
            \end{tikzpicture}
            \caption{Asym. half-bridge: utilizes only the upper half of the hysteresis curve due to non-negative magnetizing currents}
        \end{subfigure}
        \hspace{0.05\textwidth}
        \begin{subfigure}{0.45\textwidth}
            \centering
            \begin{tikzpicture}
                \tikzmath{
                    real \a, \b, \c, \d, \hn, \hc, \hm, \bc;
                    \a = 6.0;
                    \b = 1.7;
                    \c = 1.5;
                    \d = 3.0;
                    \hn = 1.25; % nominal, utilized H field value
                    \hm = 7.0; % maximal H field value
                    \hc = -\hn + 2*\c/\b; % hyteresis return H value for nominal H field
                    \bc = \a/(1 + exp(-\b*\hc+\c))-\d; % hyteresis return B value for nominal H field
                }
                \begin{axis}[very thick,
                             samples = 100,
                             xlabel = $H$,
                             ylabel = $B$,
                             xmin = -\hm,
                             xmax = \hm,
                             ymin = -4,
                             ymax = 4,
                             axis x line = middle,
                             axis y line = middle,
                             ticks = none,
                             width=\textwidth]
                    \addplot[shadecolor, dashed, domain = -\hm:\hm] {\a/(1 + exp(-\b*x+\c))-\d};
                    \addplot[shadecolor, dashed ,domain = -\hm:\hm] {\a/(1 + exp(-\b*x-\c))-\d};
                    \addplot[signaldelta, name path=A, domain=\hc:\hn, samples = 100] {\a/(1 + exp(-\b*x+\c))-\d};
                    \addplot[signaldelta, name path=B, domain=-\hn:-\hc, samples = 100] {\a/(1 + exp(-\b*x-\c))-\d};
                    \addplot[shadecolor, opacity=0.3] fill between[of=A and B];
                    \addplot[signaldelta] coordinates {(-\hc, -\bc) (\hn, -\bc)};
                    \addplot[signaldelta] coordinates {(-\hn, \bc) (\hc, \bc)};
                \end{axis}
            \end{tikzpicture}
            \caption{Full-bridge: utilizes both positive and negative hysteresis curve parts due the four-quadrant input stage}
        \end{subfigure}
            \caption{Hysteresis curves of the forward converter's transformer with different input stages (qualitative and simplified representation)}
        \end{figure}
\end{frame}

%%%%%%%%%%%%%%%%%%%%%%%%%%%%%%%%%%%%%%%%%%%%%%%%%%%%%%%%%%%%%
%% Forward converter with full-bridge input stage %%
%%%%%%%%%%%%%%%%%%%%%%%%%%%%%%%%%%%%%%%%%%%%%%%%%%%%%%%%%%%%%
\begin{frame}
    \frametitle{Forward converter with full-bridge input stage}
    The average input voltage $\overline{u}_\mathrm{p}$ of the full-bridge forward converter is conceptually identical to the asym. half-bridge variant and with the constraint
    \begin{equation*}
        \overline{u}_\mathrm{L_\mathrm{m}} = \frac{1}{T_\mathrm{s}}\int_{0}^{T_\mathrm{s}}u_\mathrm{p}(t)\mathrm{d}t = 0
    \end{equation*}
    the duty cycle also remains limited to
    $$
    D\leq\frac{1}{2} .
    $$\pause
    However, the full-bridge realization comes with distinct differences compared to the asym. half-bridge:
    \begin{itemize}
        \item Utilizes magnetic core more efficiently, i.e., core can be made smaller or less winding turns are required. \pause
        \item Effective switching frequency is doubled allowing for smaller filter components. \pause
        \item Obvious disadvantage: more complex input stage (costs).
    \end{itemize}
\end{frame}

%%%%%%%%%%%%%%%%%%%%%%%%%%%%%%%%%%%%%%%%%%%%%%%%%%%%%%%%%%%%%
%% Forward converter with additional demagnetization winding %%
%%%%%%%%%%%%%%%%%%%%%%%%%%%%%%%%%%%%%%%%%%%%%%%%%%%%%%%%%%%%%
\begin{frame}[b]
    \frametitle{Forward converter with additional demagnetization winding}
    Alternative: \hl{transfer the idea of the flyback converter} and add another winding $N_3$ to the transformer with reversed polarity. When $T$ blocks, the energy stored in the transformer's magnetic field is inherited by $N_3$ and transferred back to the input. 
    \begin{figure}
        \begin{circuitikz}[]
            %primary side
            \draw (0,0) to [short, i=$i_1(t)$] ++(1.5,0) -- ++(1.0,0) 
            to [inductor, n=l1] ++(0,-2) 
            to [Tnpn, n=npn1, invert] ++(0,-2) coordinate (C) -- (0,-4.0) 
            (0,0) to [open, o-o, v_= $u_1(t)$, voltage = straight] ++(0,-4.0);
            %secondary side
            \draw  (4.25,0) to [inductor, n=l2, mirror] ++(0,-2) 
            to [short] ++(2,0) coordinate (A)
            to [diode, l_=$D_2$, *-*, v^<= $u_\mathrm{s}$, voltage = straight] ++(0,2) coordinate (B)
            to [inductor, l=$L$] ++(2,0)
            to [short, i=$i_2(t)$] ++(0.75,0)
            to [open, o-o, v^= $u_2(t)$, voltage = straight] ++(0,-2)
            to [short] (A)
            (4.25,0) to [diode, l=$D_1$] (B);
            %demag winding
            \draw  (3.5,0) to [inductor, n=l3, mirror] ++(0,-2) 
            to [short, i<=$i_3(t)$] ++(0,-2.0)
            to [short, -*] (C)
            (3.5,0) -- ++(0,1)
            to [diode, l=$D_3$] ++(-2,0)
            to [short, -*] ++(0,-1);    
            % misc / labels
            \draw (1.5,-0.25) to [open, v = $u_\mathrm{p}$, voltage = straight] ++(0,-1.5);
            \draw let \p1 = (npn1.B) in node[anchor=east] at (\x1,\y1) {$T$};
            \path (l1.ul dot) node[circ]{}
                  (l2.ul dot) node[circ]{}
                  (l3.ur dot) node[circ]{}
                  (l3.ul dot) node{$N_3$};
            \draw (l1.midtap) node[left]{$N_1$}
            (l2.midtap) node[right]{$N_2$};
            \draw[double, double distance=3pt, thick] let \p1=(l1.core west), \p2=(l3.core east) in (\x1/2+\x2/2, \y1) -- (\x1/2+\x2/2, \y2);
            \draw[double, double distance=3pt, thick, xshift=5mm] let \p1=(l3.core west), \p2=(l2.core east), \p3=(l3.midtap) in (\x3/2+\x2/2, \y1) -- (\x3/2+\x2/2, \y2);
        \end{circuitikz}
        \caption{Forward converter with demagnetization winding (aka \hl{single-ended forward converter})}
        \label{fig:forward_converter_demagnetization_winding}
    \end{figure}
\end{frame}

%%%%%%%%%%%%%%%%%%%%%%%%%%%%%%%%%%%%%%%%%%%%%%%%%%%%%%%%%%%%%
%% Forward converter: steady-state time-domain behavior (demag. winding) %%
%%%%%%%%%%%%%%%%%%%%%%%%%%%%%%%%%%%%%%%%%%%%%%%%%%%%%%%%%%%%%
\begin{frame}[fragile]
    \frametitle{Forward converter: steady-state time-domain behavior (demag. winding)}
    \begin{figure}
        \begin{tikzpicture}
            \pgfmathsetmacro{\D}{0.35} % duty cycle
            \begin{groupplot}[group style={group size=1 by 3, xticklabels at = edge bottom}, height=0.34\textheight, width=0.875\textwidth, xmin=0, xmax=4, grid,clip = false, ymin = 0, ymax =1.1]

                % Top plot: input voltage a primary side
                \nextgroupplot[ylabel = {$u_\mathrm{p}(t)$}, ytick = {-1, 0, 1}, yticklabels = {$-U_1$, 0, $U_1$}, ymin = -1.1, ymax =1.1]
                    \pgfplotsinvokeforeach{0,...,3}{
                        \edef\AddPlot{\noexpand\addplot[signalalpha, thick] coordinates {({0 + #1},0) ({0 + #1},1) ({\D + #1},1) ({\D + #1},-0.75) ({\D+\D/0.75 + #1},-0.75) ({\D+\D/0.75 + #1},0) ({1 + #1},0)};}
                        \AddPlot
                    }
                    \draw [thick,<->]  (0,0) -- node[above]{$D T_\mathrm{s}$}(\D, 0);  
                    \draw [thick,<->]  (\D,0) -- node[above]{$T_\mathrm{m}$}(\D+\D/0.75, 0); 
                    \draw [thick,<->]  (0.0,-1.45) -- node[below]{$T_\mathrm{s}$}(1.0, -1.45); 
                    \draw[{Latex[length=2mm]}-, thin] (axis cs:1+\D+\D/0.75/2.5,-0.70) -- node[right=1mm, fill=white, inner sep = 1pt, anchor=south]{$-U_1\frac{N_1}{N_3}$} (axis cs:1+\D+\D/0.75/2.5,0.4); % indicate voltage during demag time interval


                % Middle plot: switch voltage at buck side
                \nextgroupplot[ylabel = {$u_\mathrm{s}(t)$}, ytick = {0, 0.5, 1}, yticklabels = {0, , $\frac{N_2}{N_1}U_1$}]
                    \pgfplotsinvokeforeach{0,...,3}{
                        \edef\AddPlot{\noexpand\addplot[signalalpha, thick] coordinates {({0 + #1},0) ({0 + #1},1) ({\D + #1},1) ({\D + #1},0) ({1 + #1},0) ({1 + #1},1)};}
                        \AddPlot
                    }
                    \draw[signalalpha, thick, dashed] (axis cs:0, \D) -- (axis cs:4, \D); % dashed line at U_2 (average)
                    \node[above, inner sep = 2pt, anchor = south] at (axis cs:1.5+\D/2, \D) {$U_2$}; % label U_2
                    \draw [thick,<->]  (0.0,-0.2) -- node[below]{$T_\mathrm{on}$}(\D, -0.2);  
                    \draw [thick,<->]  (\D,-0.2) -- node[below]{$T_\mathrm{off}$}(1, -0.2);  
    
                
                % Bottom plot: input current
                \nextgroupplot[ylabel = {$i_1(t)$}, xlabel={$t/T_\mathrm{s}$}, ytick = {0, 0.5, 1}, yticklabels = {0, ,}, ymin = -0.25, ymax =1.1]
                    \pgfplotsinvokeforeach{0,...,3}{
                        \edef\AddPlot{\noexpand\addplot[signaldelta, thick] coordinates {({0 + #1},0.25) ({\D + #1},0.9) ({\D + #1},-0.15) ({\D+\D/0.75 + #1},0) ({1 + #1},0) ({1 + #1},0.25)};}
                        \AddPlot
                        \edef\AddPlot{\noexpand\addplot[signaldelta, thick, name path=A] coordinates {({0 + #1},0.25) ({\D + #1},0.9)};}
                        \AddPlot
                        \edef\AddPlot{\noexpand\addplot[signaldelta, dashed, name path=B] coordinates {({0 + #1},0.25) ({\D + #1},0.7)};}
                        \AddPlot
                        \edef\AddPlot{\noexpand\addplot[signaldelta, dashed, name path=C] coordinates {({\D + #1},-0.15) ({\D+\D/0.75 + #1},0)};}
                        \AddPlot
                        \path [name path=D] ({\D + #1},0) -- ({\D+\D/0.75 + #1},0);
                        \addplot[shadecolor, opacity=0.3] fill between[of=A and B];
                        \addplot[shadecolor, opacity=0.3] fill between[of=C and D];
                    }
                    \draw[{Latex[length=2mm]}-, thin] (axis cs:\D-0.05,0.7) -- node[right=1mm, fill=white, inner sep = 1pt, anchor = west]{$i_\mathrm{m}(t)$} (axis cs:\D+0.2,0.4); % indicate mag current
                    \draw[{Latex[length=2mm]}-, thin] (axis cs:1+\D+0.05,-0.1) -- node[right=1mm, fill=white, inner sep = 1pt, anchor=south west]{$-i_3(t)$} (axis cs:1+\D+0.3,0.4); % indicate i3 current
            \end{groupplot}
        \end{tikzpicture}
    \end{figure}
\end{frame}

%%%%%%%%%%%%%%%%%%%%%%%%%%%%%%%%%%%%%%%%%%%%%%%%%%%%%%%%%%%%%
%% Forward converter with additional demagnetization winding (cont.) %%
%%%%%%%%%%%%%%%%%%%%%%%%%%%%%%%%%%%%%%%%%%%%%%%%%%%%%%%%%%%%%
\begin{frame}
    \frametitle{Forward converter with additional demagnetization winding (cont.)}
    The maximum magnetizing current is
    \begin{equation}
        \max\{i_\mathrm{m}(t)\} = i_\mathrm{m}(t=(k+D)T_\mathrm{s}) = \frac{U_1}{L_\mathrm{m}}DT_\mathrm{s}
    \end{equation}
    which is reached at the end of the turn-on time $T_\mathrm{on}$.\pause After switching off the transistor, the winding $N_3$ takes over the magnetizing current leading to
    \begin{equation}
        \max\{|i_\mathrm{3}(t)|\} = |i_3(t=(k+D)T_\mathrm{s})| = \frac{N_1}{N_3}\max\{i_\mathrm{m}(t)\} = \frac{N_1}{N_3}\frac{U_1}{L_\mathrm{m}}DT_\mathrm{s}.
    \end{equation}\pause
    To ensure that $i_\mathrm{m}(t=kT_\mathrm{s})=0$ holds at the next switch-on event, the voltage balance regarding the magnetizing inductance must be zero:
    \begin{equation}
        \overline{u}_\mathrm{L_\mathrm{m}} = \frac{1}{T_\mathrm{s}}\int_{0}^{T_\mathrm{s}}u_\mathrm{p}(t)\mathrm{d}t = U_1 D T_\mathrm{s} - \frac{N_1}{N_3} U_1  T_\mathrm{m} =0 .
    \end{equation}\pause
    Here, $T_\mathrm{m}$ denotes the \hl{demagnetization time interval} which results in 
    \begin{equation}
        T_\mathrm{m} = \frac{N_3}{N_1}DT_\mathrm{s}.
        \label{eq:forward_converter_demagnetization_time_interval}
    \end{equation}
\end{frame}

%%%%%%%%%%%%%%%%%%%%%%%%%%%%%%%%%%%%%%%%%%%%%%%%%%%%%%%%%%%%%
%% Forward converter with additional demagnetization winding (cont.) %%
%%%%%%%%%%%%%%%%%%%%%%%%%%%%%%%%%%%%%%%%%%%%%%%%%%%%%%%%%%%%%
\begin{frame}
    \frametitle{Forward converter with additional demagnetization winding (cont.)}
    Since the transistor switch-on time already covers $D T_\mathrm{s}$, the demagnetization time interval $T_\mathrm{m}$ is limited to
    \begin{equation}
        T_\mathrm{m} \leq (1-D)T_\mathrm{s}.
        \label{eq:forward_converter_demagnetization_time_interval_threshold}
    \end{equation}\pause
    Combining \eqref{eq:forward_converter_demagnetization_time_interval} and \eqref{eq:forward_converter_demagnetization_time_interval_threshold} yields
    \begin{equation}
        \frac{N_3}{N_1} \leq \frac{1-D}{D} \quad \Leftrightarrow \quad D \leq \frac{N_1}{N_1+N_3}
        \label{eq:forward_converter_demagnetization_turns_ratio_threshold}
    \end{equation}
    as a \hl{threshold for the turns ratio} to enable certain switch-on times.\pause Also, it should be noted that the turns ratio directly influences the \hl{maximum blocking voltage of the transistor}:
    \begin{equation}
        \max\{u_\mathrm{T}(t)\} = U_1 + U_1 \frac{N_1}{N_3} = U_1 \left(1 + \frac{N_1}{N_3}\right). 
    \end{equation}\pause
    Hence, to allow relatively high duty cycles by a high $N_1$ to $N_3$ ratio, cf. \eqref{eq:forward_converter_demagnetization_turns_ratio_threshold}, the blocking voltage of the transistor increases.
\end{frame}

%%%%%%%%%%%%%%%%%%%%%%%%%%%%%%%%%%%%%%%%%%%%%%%%%%%%%%%%%%%%%
%% Section summary %%
%%%%%%%%%%%%%%%%%%%%%%%%%%%%%%%%%%%%%%%%%%%%%%%%%%%%%%%%%%%%%
\begin{frame}
    \frametitle{Section summary}
    This section provided a (very) limited introduction to isolated DC-DC converters with the forward and flyback converters as examples.\pause The key takeaways are:
    \begin{itemize}
        \item The forward converter is a buck-derived topology while the flyback converter is a buck-boost-derived topology.\pause
        \item A transformer is used to provide galvanic isolation between input and output.\pause
        \item Limiting the magnetiziation of the transformer is a key aspect in the operation of these converters to prevent saturation (nonlinear behavior, extra losses).
    \end{itemize}\pause
    In addition, there are many other isolated topologies that are used in practice, e.g., 
    \begin{itemize}
        \item Push-pull converter,
        \item Isolated Ćuk / SEPIC variants, 
        \item Boost-derived topologies with full-/half bridge input stages,
        \item ...
    \end{itemize}
\end{frame}