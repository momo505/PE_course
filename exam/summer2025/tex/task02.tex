%%%%%%%%%%%%%%%%%%%%%%%%%%%%%%%%%%%%%%%%%%%%%%%%%%%%%%%%%%%%%
%% Examtask 2: Four quadrant converter with pulse width modulation %%
%%%%%%%%%%%%%%%%%%%%%%%%%%%%%%%%%%%%%%%%%%%%%%%%%%%%%%%%%%%%%

\task{Multi-port (flyback) converter for the production area}

\taskGerman{Multiport-(Flyback-)Konverter für den Produktionbereich}

% Insipired by Task 2 of the winter 2023/24 exam of Prof. Pacas (cf. \\filer.uni-siegen.de\NAS\IAS\300_Teaching\359_Lecture_notes_Prof_Pacas\Leistungselektronik\Alte Klausuren Leistungselektronik)
% However, there should be some variations to the original task:
% - Add the information that the PWM is interleaving (while not providing a second reference or carrier signal, as this should 
%   be identifed by the students when it comes to the switching pattern) and also not providing a switching table or something
% - add an additional question asking for the conducting / current carrying components during each sub switching interval 
%   (could be added to the plot), that is which transistors/diodes carry the current at each time step
% - change the values of the voltage, inductor and initial current conditions
% - change the nomenclature of the task to the usual definitions of our course

In a modern industrial automation system, a central $\SI{12}{\volt}$ DC power supply is used to power distributed sensor units and actuators throughout a production area.
While many actuators and some programmable logic controllers (PLC modules) operate directly on $\SI{12}{\volt}$, some sensors (e.g., proximity switches, temperature sensors, 
camera modules) require a stable voltage of $\SI{5}{\volt}$ or $\SI{3.3}{\volt}$. Therefore, a multi-port (flyback) converter with the parameters according 
\autoref{table:MultiportFlybackConverter} is used.

\begin{germanblock}
In einem modernen industriellen Automatisierungssystem wird eine zentrale Gleichstromversorgung mit $\SI{12}{\volt}$ verwendet, um verteilte Sensoreinheiten und Aktoren
 in einem Produktionsbereich mit Strom zu versorgen. Während viele Aktoren und einige kleine speicherprogrammierbare Steuerungen (SPS-Module) direkt mit $\SI{12}{\volt}$ arbeiten, 
benötigen manche Sensoren (z. B. Näherungsschalter, Temperatursensoren, Kameramodule) eine stabile Spannung von $\SI{5}{\volt}$ oder $\SI{3.3}{\volt}$. Dazu wird ein Multi-Port Flybackkonverter
mit den in \autoref{table:MultiportFlybackConverter} angegebenen Parametern verwendet.
\end{germanblock}

% figure figStepDownConverterOutputFilter
%%%%%%%%%%%%%%%%%%%%%%%%%%%%%%%%%%%%%%%%%%%%%%%%%%%%%%%%%%%%%
%% Outlook: multi-port (flyback) converter %%
%%%%%%%%%%%%%%%%%%%%%%%%%%%%%%%%%%%%%%%%%%%%%%%%%%%%%%%%%%%%%

\begin{figure}[ht]
    \centering
    \begin{tikzpicture}
        \draw (-0.5,2) to[inductor, name=l1] ++(0,-2)
        (-0.5,2) to [short, -*, i<_=$i'_1(t)$] ++(-1,0) coordinate (A1)
        (-0.5,0) to [short, -*] ++(-1,0) coordinate (A2)
        (A1) to [inductor, l_=$L_\mathrm{m}$, i=$i_\mathrm{m}(t)$] (A2)
        (A1) to [short, -o, i<_=$i_1(t)$] ++(-1.5,0) coordinate (A11)
        % Add transistor
        (A2) to [Tnpn, n=npn1, invert] ++(0,-3) coordinate (Agnd)
        (Agnd) to [short, -o] ++(-1.5,0) coordinate (Aggnd)
        (A11) to [open, v=$u_1$, voltage = straight] (Aggnd);


        \draw (0.5,2) to[inductor, name=l2, mirror] ++(0,-2)
        (0.5,2) to [short, -o, i=$i_2(t)$] ++(1,0) coordinate (B1)
        (0.5,0) to [short, -o] ++(1,0) coordinate (B2)
        (B1) to [open, v^=$u_2$, voltage = straight] (B2);
        \draw (0.5,-1) to[inductor, name=l3, mirror] ++(0,-2)
        (0.5,-1) to [short, -o, i=$i_3(t)$] ++(1,0) coordinate (C1)
        (0.5,-3) to [short, -o] ++(1,0) coordinate (C2)
        (C1) to [open, v^=$u_3$, voltage = straight] (C2);
        \draw[double, double distance=3pt, thick] let \p1=(l1.core west), \p2=(l3.core east) in (\x1/2+\x2/2, \y1) -- (\x1/2+\x2/2, \y2);
        \path (l1.ul dot) node[circ]{}
        (l2.ul dot) node[circ]{}
        (l3.ul dot) node[circ]{};
        \draw (l1.midtap) node[left]{$N_1$}
        (l2.midtap) node[right]{$N_2$}
        (l3.midtap) node[right]{$N_3$};

        % Add 5V devices on the right side
        \draw (B1) to [short, o-] ++(2 ,0) coordinate (devp5V)
        (B2) to [short, o-] ++(2,0) coordinate (devm5V)
        (devp5V) to [R,l_=$R$,v^=$U_\text{2}$,voltage shift=0.5, voltage=straight] (devm5V)
        (B2) ++(1.2,1) coordinate (rectdevm5V);
        \node[draw, dashed, minimum width=2cm, minimum height=2.3cm, anchor=west, label=below:$\SI{5}{\volt}$-Devices] at (rectdevm5V) (dev) {};
        % Add 3.3V devices on the right side
        \draw (C1) to [short, o-] ++(2 ,0) coordinate (devp3_3V)
        (C2) to [short, o-] ++(2,0) coordinate (devm3_3V)
        (devp3_3V) to [R,l_=$R$,v^=$U_\text{3}$,voltage shift=0.5, voltage=straight] (devm3_3V)
        (C2) ++(1.2,1) coordinate (rectdevm3_3V);
        \node[draw, dashed, minimum width=2cm, minimum height=2.3cm, anchor=west, label=below:$\SI{3.3}{\volt}$-Devices] at (rectdevm3_3V) (dev) {};


    \end{tikzpicture}
    \caption{Multi-port (flyback) converter.}
    \label{fig:MultiPortFlybackConverter}
\end{figure}


\begin{table}[ht]
    \centering  % Zentriert die Tabelle
    \begin{tabular}{llll}
        \toprule
        Input voltage: &  $U_{\mathrm{1}} = \SI{12}{\volt}$ & Switching frequency: & $f_\mathrm{s} = \SI{100}{\kilo\hertz}$\\ 
        Output voltage: & $U_{\mathrm{2}} = \SI{5}{\volt}$  & Min. output current: & $I_{\mathrm{2,min}} = \SI{2}{\ampere}$ \\ 
        Output voltage: &  $U_{\mathrm{3}} = \SI{3.3}{\volt}$ & Min. output current: & $I_{\mathrm{3,min}} = \SI{3}{\ampere}$ \\ 
        Duty cycle: & 0.2 & Primary turns & $N_{\mathrm{1}} = 2040$\\ 
        \midrule
        \multicolumn{4}{l}{\quad All components are ideal. No losses need to be considered.}  \\ 
        \bottomrule
    \end{tabular}
    \caption{Parameters of the multi-port flyback converter.}  % Beschriftung der Tabelle
    \label{table:MultiportFlybackConverter}
\end{table}


\subtask{Calculate the number of turns $N_\mathrm{2}$ and $N_\mathrm{3}$ for the two outputs.}{2}
\subtaskGerman{Berechnen Sie die Anzahl der Windungen $N_\mathrm{2}$ und $N_\mathrm{3}$ für die beiden Ausgänge.}
 
% Solution of subtask
\begin{solutionblock}
  The voltage ratio depends on the duty cycle and turn ratio
  \begin{equation*}
      \frac{U_\mathrm{2}}{U_\mathrm{1}}=\frac{N_\mathrm{2}}{N_\mathrm{1}} \frac{D}{1-D}.
  \end{equation*}
  Solving the equation with respect to $N_\mathrm{2}$ yields
  \begin{equation*}
      N_\mathrm{2}=\frac{U_\mathrm{2}}{U_\mathrm{1}} N_\mathrm{1} \frac{1-D}{D} = \frac{\SI{5}{\volt}}{\SI{12}{\volt}} 2040 \frac{0.2}{1-0.2}=3400.
  \end{equation*}
  In the same $N_\mathrm{3}$ yields
  \begin{equation*}
      N_\mathrm{3}=\frac{U_\mathrm{3}}{U_\mathrm{1}} N_\mathrm{1} \frac{1-D}{D} = \frac{\SI{3.3}{\volt}}{\SI{12}{\volt}} 2040 \frac{1-0.2}{0.2}=2244.
  \end{equation*}
\end{solutionblock}

\subtask{Calculate the required magnetizing inductance $L_\mathrm{m}$ so that the multi-port flyback converter is in BCM at minimum current consumption.}{3}
\begin{hintblock}
    the input current $i_\mathrm{1}$ is composed of the transformed current components from the $\SI{5}{\volt}$ output and from the $\SI{3.3}{\volt}$ output according 
    to $i_\mathrm{1}=i_\mathrm{1,port1} + i_\mathrm{1,port2}$.   
\end{hintblock}
\subtaskGerman{Berechnen Sie die erforderliche Magnetisierungsinduktivität $L_\mathrm{m}$, wenn sich der Multiport-Flyback-Konverter bei minimalem Stromverbrauch 
im BCM befindet.}
\begin{germanhintblock} 
    Der Eingangsstrom $i_\mathrm{1}$ setzt sich aus den transformierten Stromanteilen vom $\SI{5}{\volt}$ Ausgang und vom $\SI{3.3}{\volt}$ Ausgang 
    nach  $i_\mathrm{1}=i_\mathrm{1,port1} + i_\mathrm{1,port2}$ zusammen.
\end{germanhintblock}


% Solution of subtask
\begin{solutionblock}
    At boundary conduction mode the average of magnetizing current $i_\mathrm{Lm}$  corresponds to the half of its delta and 
    and the magnetic current can be calculated with help of the input current.
    The average input current is calculated by the sum of both output currents, which are to be transformed by the turn ratio and the duty cycle.
    For boundary conduction mode the minimum output current of each port is to apply:
    \begin{equation*}
        \overline{i}_\mathrm{1,min}=\frac{N_\mathrm{2}}{N_\mathrm{1}} \frac{1-D}{D} \overline{i}_\mathrm{2,min} + \frac{N_\mathrm{3}}{N_\mathrm{1}} \frac{1-D}{D} \overline{i}_\mathrm{3,min}
        = \frac{3400}{2040} \frac{1-0.2}{0.2} \SI{2}{\ampere} + \frac{2244}{2040} \frac{1-0.2}{0.2} \SI{3}{\ampere} = \SI{1.68}{\ampere}.
    \end{equation*}
    The relation between average of the magnetizing current $\overline{i}_\mathrm{Lm}$ and the average of the input current yields
    \begin{equation*}
        \overline{i}_\mathrm{L} \cdot T_\mathrm{on} = \overline{i}_\mathrm{1} \cdot T_\mathrm{s}.
    \end{equation*}
    The average of the magnetizing current $i_\mathrm{Lm}$ is calculated by
    \begin{equation*}
        \overline{i}_\mathrm{L} =  \frac{\overline{i}_\mathrm{1}}{D} =  \frac{\SI{1.68}{\ampere}}{0.2}= \SI{8.38}{\ampere}.
    \end{equation*}
    At boundary conduction mode the ripple of magnetizing current $i_\mathrm{Lm}$  results in
    \begin{equation*}
        \Delta i_\mathrm{Lm} =  2 \cdot \overline{i}_\mathrm{Lm} = 2 \cdot \SI{8.38}{\ampere} = \SI{16.75}{\ampere}.
    \end{equation*}
    The value of the inductance is calculate by
    \begin{equation*}
        \Delta i_\mathrm{Lm}= \frac{U_\mathrm{1}}{L} \cdot T_\mathrm{on} = \frac{U_\mathrm{1}}{L_\mathrm{m}} \cdot \frac{D}{f_\mathrm{s}}  .
    \end{equation*}
    Solving this equation with respect to $L_\mathrm{m}$ yields
    \begin{equation*}
        L_\mathrm{m}= \frac{U_\mathrm{1}}{\Delta i_\mathrm{Lm}} \cdot \frac{D}{f_\mathrm{s}} = \frac{\SI{12}{\volt}}{\SI{16.75}{\ampere}} \cdot \frac{0.2}{\SI{100}{\kilo\hertz}} = \SI{1.43}{\micro\henry}.
    \end{equation*}
\end{solutionblock}

\subtask{Calculate the output voltage $U_{\mathrm{2}}$ for the case, 
  that all loads of the $\SI{3.3}{\volt}$-output are disconnected, while minimum current $I_{\mathrm{2,min}}$ at $\SI{5}{\volt}$ port.
  What is the risk in this case?}{1}

\subtaskGerman{Berechnen Sie die Ausgangsspannung $U_{\mathrm{2}}$ für den Fall, dass alle Lasten des $\SI{3.3}{\volt}$-Ausgangs getrennt sind, 
während der minimale Strom $I_{\mathrm{2,min}}$ am $\SI{5}{\volt}$-Ausgang fließt. Was ist das Risiko in diesem Fall?}


% Solution of subtasks
\begin{solutionblock}
    The voltage in DCM-mode for an flyback converter is calculated by
    \begin{equation*}
        U_\mathrm{2} = U_\mathrm{1}^2 \frac{D^2 T_\mathrm{s}}{2 L_\mathrm{m} \overline{i}_\mathrm{2,min}} = U_\mathrm{1}^2 \frac{D^2 }{2 L_\mathrm{m} \overline{i}_\mathrm{2,min} f_\mathrm{s}} 
        = \SI{12}{\volt}^2 \frac{0.2^2 }{2 \cdot\SI{1.43}{\micro\henry} \cdot \SI{2}{\ampere} \cdot \SI{100}{\kilo\hertz} }  = \SI{10.05}{\volt}.
    \end{equation*}

    The voltage increase may leads to the damage of the $\SI{5}{\volt}$ devices.

\end{solutionblock}
